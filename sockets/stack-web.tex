
%diferencia entre texto y binario
%bin / text
%prefix / delim
\input{../preamble-tmp.tex}

\title%
{Sockets TCP/IP en C++ - Stack Web}


\subject{Sockets TCP/IP en C++ - Stack Web}


\begin{document}

% firefox + wireshark
%
% httpie -b https://www.google.com
% httpie -b http://www.google.com
% httpie -v http://www.google.com
%
% netcat www.google.com http < REQ
%
% nslookup www.google.com
%
% netcat IP 80 < REQ

~% macro content() %~
\section{Stack Web}
\subsection*{}
\begin{frame}[fragile]{Stack Web (simplificado)}
         \begin{tikzpicture}[cell/.style={rectangle,draw=black},
            space/.style={minimum height=1.5em,matrix of nodes,row sep=-\pgflinewidth,column sep=-\pgflinewidth,column 1/.style={font=\ttfamily}},text depth=0.5ex,text height=2ex,nodes in empty cells]

            \matrix (first)[space, column 1/.style={font=\ttfamily},column 2/.style={nodes={cell,minimum width=12em}},column 3/.style={nodes={cell,minimum width=2em}},column 4/.style={nodes={cell,minimum width=2em}},column 5/.style={nodes={cell,minimum width=2em}}]
            {
               {\visible<6->{}}             & {\visible<6->{...}} \\
               {\visible<6->{firefox}}      & {\visible<6->{HTML / CSS / JS}}  \\
               {\visible<5->{httpie}}       & {\visible<5->{HTTP}}   \\
               {\visible<4->{}}             & {\visible<4->{TLS}} \\
               {\visible<3->{netcat/OS}}    & {\visible<3->{TCP}} \\
               {\visible<2->{OS}}           & {\visible<2->{IP}} \\
               {\visible<2->{}}             & {\visible<2->{...}} \\
               {\visible<1->{HW}}           & {\visible<1->{Maxwell's eq}} \\
            };
         \end{tikzpicture}
\end{frame}
\note[itemize] {
\item Nuestras comunicaciones se basana en las leyes de la f\'isicas descriptas por las ecuaciones de Maxwell.
\item Trabajar con campos electromagneticos es trabajo del hardware. Sobre \'el el sistema operativo resuelve el ruteo por la red IP y el transporte de los datos via TCP.
\item TCP esta en el l\'imite entre las aplicaciones de user (\lstinline[style=normal]!netcat! y otros) y el OS: mientras que el OS implementa el protocolo TCP, la aplicaci\'on de user la usa.
\item TLS se usa para aut\'enticas y encriptar las comunicaciones (SSL fue su antecesor). Aunque es opcional, muchas aplicaciones hoy usan TLS y as\'i deber\'ia ser.
\item \lstinline[style=normal]!httpie! (\lstinline[style=normal]!curl!, \lstinline[style=normal]!wget!, \lstinline[style=normal]!aria2!) son capaces de comunicarse con servidores y hablar HTTP aunque la interpretaci\'on del contenido (HTML, CSS) es limitada o nula.
\item No tomes este diagrama literal: es una versi\'on simplificada.
}

\begin{frame}[fragile]{Resolucion de nombres (simplificado)}
         \begin{tikzpicture}[cell/.style={rectangle,draw=black},
            space/.style={minimum height=1.5em,matrix of nodes,row sep=-\pgflinewidth,column sep=-\pgflinewidth,column 1/.style={font=\ttfamily}},text depth=0.5ex,text height=2ex,nodes in empty cells]

            \matrix (first)[space, column 1/.style={font=\ttfamily},column 2/.style={nodes={cell,minimum width=12em}},column 3/.style={nodes={cell,minimum width=2em}},column 4/.style={nodes={cell,minimum width=2em}},column 5/.style={nodes={cell,minimum width=2em}}]
            {
               {\visible<4->{nslookup}}     & {\visible<4->{DNS}} \\
               {\visible<3->{nslookup/OS}}  & {\visible<3->{UDP}} \\
               {\visible<2->{OS}}           & {\visible<2->{IP}} \\
               {\visible<2->{}}             & {\visible<2->{...}} \\
               {\visible<1->{HW}}           & {\visible<1->{Maxwell's eq}} \\
            };
         \end{tikzpicture}
\end{frame}
\note[itemize] {
\item Resolver un service name a un puerto es f\'acil. Los servicios est\'andar son pocos y no cambian, un archvio con el mapping alcanza (\lstinline[style=normal]!/etc/services!)
\item Resolver un hostname a una IP es mucho m\'as complicado, un archivo (\lstinline[style=normal]!/etc/hosts!) no es lo suficientemente din\'amico.
\item El protocolo DNS se encarga de resolver un hostname a una direccion IPv4 o IPv6.
\item Al igual que con TCP, UDP est\'a en el l\'imite entre aplicaciones de user (\lstinline[style=normal]!nslookup!) y el OS.
}

\begin{frame}[fragile]{TP de Taller (simplificado)}
         \begin{tikzpicture}[cell/.style={rectangle,draw=black},
            space/.style={minimum height=1.5em,matrix of nodes,row sep=-\pgflinewidth,column sep=-\pgflinewidth,column 1/.style={font=\ttfamily}},text depth=0.5ex,text height=2ex,nodes in empty cells]

            \matrix (first)[space, column 1/.style={font=\ttfamily},column 2/.style={nodes={cell,minimum width=12em}},column 3/.style={nodes={cell,minimum width=2em}},column 4/.style={nodes={cell,minimum width=2em}},column 5/.style={nodes={cell,minimum width=2em}}]
            {
               {\visible<1->{taller}}       & {\visible<1->{...}} \\
               {\visible<1->{taller/OS}}    & {\visible<1->{TCP}} \\
               {\visible<1->{OS}}           & {\visible<1->{IP}} \\
               {\visible<1->{}}             & {\visible<1->{...}} \\
               {\visible<1->{HW}}           & {\visible<1->{Maxwell's eq}} \\
            };
         \end{tikzpicture}
\end{frame}
\note[itemize] {
\item A grandes rasgos es sobre TCP/UDP donde nos paramos y desarrollamos el TP en Taller.
}


~% endmacro %~

~{ content() }~

\end{document}
