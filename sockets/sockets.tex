\input{../preamble-tmp.tex}

\title%
{Sockets TCP/IP en C++}

\subject{Sockets TCP/IP en C++}

\usepackage{docmute}
\begin{document}

\begin{frame}[noframenumbering,plain]
   \titlepage
\end{frame}

\begin{frame}{De qu\'e va esto?}
   \tableofcontents
   % You might wish to add the option [pausesections]
\end{frame}

% Open https://www.google.com with the Firefox and sniff with Wireshark (too chatty!)
%
% Open https://www.google.com with httpie and sniff (much simpler!)
%   - show HTML
%
% The same but to http:// instead of https://
%   - show the HTTP protocol is there
%   - but why we cannot see the HTML in Wireshark? It is compressed!
%
% Run nc http://www.google.com  < req
%   - nc does not know about URLs
%
% Run nc www.google.com  http  < req
%   - it works but a lot of garbage is printed
%   - netcat does not know how to interpret the response nor how to decompress
%
% Patch req and run nc again
%
% cat /etc/services and run nc www.google.com 80 < req
%
% cat /etc/hosts but www.google.com  is not there!
% nslookup www.google.com
%
% Run nc <IP> 80 < req
%
% Run nc <IP2> 80 < req  where <IP2> is like the original IP but
% *slightly* modified. If it works talk about multiple IPs for the same hostname.
%   - this is the reason for the loop!
%
% Then try <IP2> very different, proving that it is not Google there.
%

~% from "stack-web.tex" import content as stackweb_content with context %~
~{ stackweb_content() }~

~% from "tcp_ip_network_overview.tex" import content as tcpip_content with context %~
~{ tcpip_content() }~

~% from "sockets-tcp.tex" import content as skt_content with context %~
~{ skt_content() }~

% ~#
% ~% from "protocols.tex" import content as proto_content with context %~
% ~{ proto_content() }~
% #~


\appendix
\section<presentation>*{\appendixname}
\subsection<presentation>*{Referencias}

\begin{frame}[allowframebreaks]
   \frametitle<presentation>{Referencias}

   \begin{thebibliography}{10}

         \beamertemplatearticlebibitems
         % Followed by interesting articles. Keep the list short.

      \bibitem{manpages}
         man getaddrinfo
      \bibitem{manpages}
         man netcat
      \bibitem{manpages}
         man netstat

     \bibitem{RFCs}
        RFCs 971, 2460, ...

     \bibitem{RFCs}
        RFCs 793, ...

    \bibitem{book}
        TCP/IP Illustrated, Richard Stevens

    \bibitem{book}
        Data and Computer Comunications, Ed Stallings
   \end{thebibliography}
\end{frame}

\end{document}


