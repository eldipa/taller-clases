
\input{../preamble-tmp.tex}

\title%
{Clases en C++ - RAII}

\subject{Clases en C++ - RAII}

\begin{document}

\begin{frame}[noframenumbering,plain]
   \titlepage
\end{frame}


~% macro content() %~
\section{RAII: Resource Acquisition Is Initialization}
\subsection{Constructor y destructor}
\begin{frame}[fragile]{Constructor/destructor: manejo de recursos autom\'atico}
                   ~% if interactive is off %~
        \begin{lstlisting}[style=normal,firstnumber=1,linebackgroundcolor={%
                 \btLstHLB<1>{5-8}%
                 \btLstHLB<2>{14-16}%
         }]~% else %~
        \begin{lstlisting}[style=normal,firstnumber=1]~% endif %~
struct Vector {
  int *data;
  int size;

  Vector(int size) {  // create
    this->data = malloc(size*sizeof(int));
    this->size = size;
  }

  int get(int pos) {
    return this->data[pos];
  }

  ~Vector() { // destroy
    free(this->data);
  }
};
        \end{lstlisting}
\end{frame}
\note[itemize] {
\item El constructor es un c\'odigo que se ejecuta al momento de crear un nuevo objeto. C++ siempre llama a algun constructor al crear un nuevo objeto.
\item Todos los objetos son creados por un constructor. Si un TDA no tiene un constructor, C++ crea un constructor por default.
\item Un TDA puede tener m\'ultiples constructores (que los veremos a continuaci\'on). Sin embargo s\'olo puede haber un \'unico destructor.
\item Un destructor es un c\'odigo que se ejecuta al momento de destuirse un objeto (cuando este se va de scope o es eliminado del heap con \lstinline[style=normal]!delete!).
\item Todos los objetos tienen un destructor. Si un TDA no tiene un destructor, C++ crea un destructor por default.
}

\begin{frame}[fragile]{Reduciendo la probabilidad de errores}
                   ~% if interactive is off %~
\only<1>{Diferencia entre reservar memoria y construir un objeto}
\only<2>{Memoria sin inicializar}
\only<3>{Destrucci\'on autom\'atica}
                   ~% endif %~
   \begin{columns}[t]
      \begin{column}{.4\linewidth}
                   ~% if interactive is off %~
         \begin{lstlisting}[style=normal,firstnumber=29,linebackgroundcolor={%
                 \btLstHLB<1>{31}%
                 \btLstHLR<2>{33,35}%
                 \btLstHLR<3>{38}%
         }]~% else %~
         \begin{lstlisting}[style=normal,firstnumber=29]~% endif %~
// En C
void g() {
  struct Vector v;

  v.data;

  vector_create(&v, 5);
  //...

}
         \end{lstlisting}
     \end{column}
      \begin{column}{.4\linewidth}
                   ~% if interactive is off %~
         \begin{lstlisting}[style=normal,firstnumber=29,linebackgroundcolor={%
                 \btLstHLB<1>{31}%
                 \btLstHLB<2>{31,33}%
                 \btLstHLB<3>{38}%
         }]~% else %~
         \begin{lstlisting}[style=normal,firstnumber=29]~% endif %~
// En C ++
void g() {
  Vector v(5);

  v.data;


  //...

}
         \end{lstlisting}
     \end{column}
 \end{columns}
\end{frame}
\note[itemize] {
\item Con los constructores (si estan bien escritos) no se puede usar un objeto sin inicializar.
\item Con los destructores (si estan bien escritos, se usa RAII y usamos el stack) no vamos a tener leaks.
\item Los destructores se llaman autom\'aticamente cuando el objeto se va de scope.
}

\begin{frame}[fragile]{Operadores new y delete}
                   ~% if interactive is off %~
        \begin{lstlisting}[style=normal,firstnumber=1,linebackgroundcolor={%
                 \btLstHLB<1>{6,15}%
         }]~% else %~
        \begin{lstlisting}[style=normal,firstnumber=1]~% endif %~
struct Vector {
  int *data;
  int size;

  Vector(int size) {
    this->data = new int[size]();
    this->size = size;
  }

  int get(int pos) {
    return this->data[pos];
  }

  ~Vector() {
    delete[] this->data;
  }
};
        \end{lstlisting}
\end{frame}
\note[itemize] {
\item Las funciones \lstinline[style=normal]!malloc! y \lstinline[style=normal]!free! reservan y liberan memoria pero no crean objetos (no llaman a los constructores ni los destruyen)
\item El operador \lstinline[style=normal]!new! y su contraparte \lstinline[style=normal]!delete! no s\'olo manejan la memoria del heap sino que tambi\'en llaman al respectivo constructor y destructor.
\item Para crear un array de objetos hay que usar los operadores \lstinline[style=normal]!new[]! y \lstinline[style=normal]!delete[]! y la clase a instanciar debe tener un constructor sin par\'ametros.
\item Hay una sutil diferencia sint\'actica cuando de tipos primitivos se trata, como \lstinline[style=normal]!int! o \lstinline[style=normal]!char!. La expresi\'on \lstinline[style=normal]!new int! crea un \lstinline[style=normal]!int! sin inicializar mientras que \lstinline[style=normal]!new int()! lo inicializa a cero.
}

\subsection{Manejo de errores}
\begin{frame}[plain,fragile]{Manejo de errores en C (madness)}
        \begin{lstlisting}[style=normal]
int process() {
  char *buf = malloc(sizeof(char)*20);

  FILE *f = fopen("data.txt", "rt");
  if (!f) { free(buf); return -1;}

  int s = fread(buf, sizeof(char), 20, f);

  if (s != 20) {
    fclose(f);
    free(buf);
    return -1;
  }

  fclose(f);
  free(buf);
  return 0;
}
        \end{lstlisting}
\end{frame}
\note[itemize] {
\item En C hay que chequear los valores de retorno para ver si hubo un error o no.
\item En caso de error se suele abortar la ejecuci\'on de la funci\'on actual requiriendo previamente liberar los recursos adquiridos
\item El problema esta en que es muy f\'acil equivocarse y liberar un recurso aun no adquirido u olvidarse de liberar un recurso que si lo fue.
\item No s\'olo es una cuesti\'on de leaks de memoria. Datos corruptos por archivos o sockets mal cerrados o leaks en el sistema operativo son otros factores que no se solucionan simplemente con un garbage collector ni reiniciando el programa.
}


\begin{frame}[fragile]{Aplicaci\'on del idiom RAII}
        \begin{lstlisting}[style=normal]
struct Buffer {
  Buffer(int size) {
    this->data = new char[size];
  }
  ~Buffer() {
    delete[] this->data;
  }
};

struct File {
  File(const char *name, const char *flags) {
    this->f = fopen(name, flags);
    if (!this->f) throw std::exception("fopen failed");
  }
  ~File() {
    fclose(this->f);
  }
};
        \end{lstlisting}
\end{frame}
\note[itemize] {
\item La idea es simple, si hay un recurso (memoria en el heap, un archivo, un socket) hay que encapsular el recurso en un objeto de C++ cuyo constructor lo adquiera e inicialize y cuyo destructor lo libere.
\item N\'otese como la clave esta en el dise\~no sim\'etrico del par constructor-destructor.
\item Vamos a refinar el concepto RAII en las pr\'oximas clases con el concepto de excepciones.
}


\begin{frame}[fragile]{RAII + Stack: No leaks}
        \begin{lstlisting}[style=normal]
int process() {
  Buffer buf(20);

  File f("data.txt", "rt");
  int s = f.read(buf, sizeof(char), 20, f);

  if (s != 20)
    return -1;

  return 0;
} // <-- ~File()
  //     ~Buffer()
        \end{lstlisting}
\end{frame}
\note[itemize] {
\item Tomese un minuto para reflexionar. Vea el c\'odigo y comparelo con otros c\'odigos de otras personas o incluso de usted mismo. Es la difencia entre alguien que sabe C++ de alguien que escribe c\'odigo que compila.
\item Al instanciarse los objetos RAII en el stack, sus constructores adquieren los recursos autom\'aticamente.
\item Al irse de scope cada objeto se les invoca su destructor autom\'aticamente y por ende liberan sus recursos sin necesidad de hacerlo expl\'icitamente.
\item El c\'odigo C++ se simplifica y se hace m\'as robusto a errores de programaci\'on:  RAII + Stack es uno de los conceptos claves en C++.
\item Veremos mas sobre RAII, manejo de errores y excepciones en C++ en las pr\'oximas clases.
}

~% endmacro %~

~{ content() }~

\end{document}


