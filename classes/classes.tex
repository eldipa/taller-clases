
\pdfminorversion=4

~% if handout is on %~
    \documentclass[professionalfonts,handout]{beamer}
    \setbeameroption{show notes}
    \setbeamertemplate{note page}{\insertnote}
    \setbeamercolor{background canvas}{bg=white}
~% else %~
    ~% if interactive is on %~
        \documentclass[professionalfonts,handout]{beamer}
    ~% else %~
        \documentclass[professionalfonts]{beamer}
    ~% endif %~
~% endif %~



% Template original the Beamer: Copyright 2004 by Till Tantau <tantau@users.sourceforge.net>.


\mode<presentation>
{
  \usetheme{metropolis}
  % or ...

  %\setbeamercovered{transparent} %hace que lo que esta covered se muestre transparente
}

\setbeamertemplate{navigation symbols}{}%remove navigation symbols

\usepackage[english]{babel}

\usepackage[latin1]{inputenc}

\usepackage{times}
\usepackage[T1]{fontenc}

% para poder escribir codigo fuente en las diapositivas
\usepackage{listings}

% para graficar diagramas simples y arboles de jerarquias
\usepackage{tikz}
\usepackage{tikz-qtree}
\usetikzlibrary{matrix,positioning}
\usetikzlibrary{shapes,arrows,chains,calc,decorations.pathmorphing,decorations.shapes}

\definecolor{myLightGray}{RGB}{191,191,191}
\definecolor{myGray}{RGB}{160,160,160}
\definecolor{myDarkGray}{RGB}{144,144,144}
\definecolor{myDarkRed}{RGB}{167,114,115}
\definecolor{myRed}{RGB}{255,58,70}
\definecolor{myGreen}{RGB}{0,255,71}

\usepackage{xcolor}
\definecolor{darkgreen}{rgb}{0,.5,0}
\definecolor{darkblue}{rgb}{0,0,.7}
\lstdefinestyle{normal}{language=C++,       % lenguaje C++
   numbers=left,                            % enumerar las lineas
   keywordstyle=\color{darkblue}\textbf,    % color de las keywords
   stringstyle=\color{red},                 % color de los strings
   commentstyle=\color{darkgreen},          % color de los comentarios
   basicstyle=\color{black}\ttfamily\footnotesize\bfseries,     % color del texto en general
   morecomment=[l][\color{magenta}]{\#},    % coloreamos las intrucciones del precompilador (todo lo que empieza con #)
   ndkeywords={NULL,nullptr,siz,zer,mov,add,noexcept},               % definimos una nuevas keywords como NULL y nullptr
   ndkeywordstyle=\color{violet},           % y las nuevas keywords tendran este color
   frame=simple,                            % simple, sin ningun marco o frame alrededor del codigo
   basewidth={0.55em,0.55em}                % tamano de las letras/lineas. usado para reducir el whitespace entre estas
}

\lstdefinestyle{normal33}{language=C++,       % lenguaje C++
   numbers=left,                            % enumerar las lineas
   keywordstyle=\color{darkblue}\textbf,    % color de las keywords
   stringstyle=\color{red},                 % color de los strings
   commentstyle=\color{darkgreen},          % color de los comentarios
   basicstyle=\color{black}\ttfamily\footnotesize\bfseries,     % color del texto en general
   morecomment=[l][\color{magenta}]{\#},    % coloreamos las intrucciones del precompilador (todo lo que empieza con #)
   ndkeywords={NULL,nullptr},               % definimos una nuevas keywords como NULL y nullptr
   ndkeywordstyle=\color{violet},           % y las nuevas keywords tendran este color
   frame=simple,                            % simple, sin ningun marco o frame alrededor del codigo
   basewidth={0.55em,0.55em}                % tamano de las letras/lineas. usado para reducir el whitespace entre estas
}
%mismo estilo, sin numeros
\lstdefinestyle{normalnonumbers}{language=C++,
   keywordstyle=\color{darkblue}\textbf,
   stringstyle=\color{red},
   commentstyle=\color{darkgreen},
   basicstyle=\color{black}\ttfamily\footnotesize\bfseries,
   morecomment=[l][\color{magenta}]{\#},
   ndkeywords={NULL,nullptr,move,add},
   ndkeywordstyle=\color{violet},
   frame=simple,
   basewidth={0.55em,0.55em}
}


% mismo estilo pero todos los colores estan mas debiles como si se volvieran transparentes
% usado para poder resaltar el codigo.
\lstdefinestyle{dimmided}{language=C++,
   keywordstyle=\color{darkblue!30}\textbf,
   stringstyle=\color{red!30},
   commentstyle=\color{darkgreen!30},
   basicstyle=\color{black!30}\ttfamily\footnotesize\bfseries,
   morecomment=[l][\color{magenta!30}]{\#},
   ndkeywords={NULL,nullptr,siz,zer,mov,add},
   ndkeywordstyle=\color{violet!30},
   moredelim=**[is][\only<+->{\color{black}\lstset{style=normal}}]{@}{@}, % definimos que las lineas entre arrobas (@) tendran el estilo normal (con los colores a toda intensidad)
   frame=simple,
   basewidth={0.55em,0.55em}
}
\lstdefinestyle{dimmided42}{language=C++,
   keywordstyle=\color{darkblue!30}\textbf,
   stringstyle=\color{red!30},
   commentstyle=\color{darkgreen!30},
   basicstyle=\color{black!30}\ttfamily\footnotesize\bfseries,
   morecomment=[l][\color{magenta!30}]{\#},
   ndkeywords={NULL,nullptr,siz,zer,mov,add},
   ndkeywordstyle=\color{violet!30},
   moredelim=**[is][{\color{black}\lstset{style=normal}}]{@}{@}, % definimos que las lineas entre arrobas (@) tendran el estilo normal (con los colores a toda intensidad)
   frame=simple,
   basewidth={0.55em,0.55em}
}

\lstdefinelanguage{json}{
    literate=
     *{:}{{{\color{violet}{:}}}}{1}
      {,}{{{\color{violet}{,}}}}{1}
      {\{}{{{\color{darkgreen}{\{}}}}{1}
      {\}}{{{\color{darkgreen}{\}}}}}{1}
      {[}{{{\color{darkgreen}{[}}}}{1}
      {]}{{{\color{darkgreen}{]}}}}{1},
}


\lstdefinestyle{normaljson}{language=json,  % lenguaje json
   numbers=left,                            % enumerar las lineas
   keywordstyle=\color{darkblue}\textbf,    % color de las keywords
   stringstyle=\color{red},                 % color de los strings
   commentstyle=\color{red},                % color de los comentarios
   basicstyle=\color{black}\ttfamily\footnotesize\bfseries,     % color del texto en general
   morecomment=[l][\color{magenta}]{\#},    % coloreamos las intrucciones del precompilador (todo lo que empieza con #)
   ndkeywords={NULL,nullptr},               % definimos una nuevas keywords como NULL y nullptr
   ndkeywordstyle=\color{violet},           % y las nuevas keywords tendran este color
   frame=simple,                            % simple, sin ningun marco o frame alrededor del codigo
   basewidth={0.55em,0.55em}                % tamano de las letras/lineas. usado para reducir el whitespace entre estas
}

\lstdefinelanguage{http}{
    morekeywords={
        GET,
        PUT,
        POST,
        DELETE
    }
}

\lstdefinestyle{normalhttp}{language=http,  % lenguaje HTTP
   numbers=left,                            % enumerar las lineas
   keywordstyle=\color{darkblue}\textbf,    % color de las keywords
   stringstyle=\color{red},                 % color de los strings
   commentstyle=\color{red},                % color de los comentarios
   basicstyle=\color{black}\ttfamily\footnotesize\bfseries,     % color del texto en general
   morecomment=[l][\color{magenta}]{\#},    % coloreamos las intrucciones del precompilador (todo lo que empieza con #)
   ndkeywords={NULL,nullptr},               % definimos una nuevas keywords como NULL y nullptr
   ndkeywordstyle=\color{violet},           % y las nuevas keywords tendran este color
   frame=simple,                            % simple, sin ningun marco o frame alrededor del codigo
   basewidth={0.55em,0.55em}                % tamano de las letras/lineas. usado para reducir el whitespace entre estas
}

% Pone las constantes numericas con color (violeta en este caso):
% - los numeros que estan dentro de un string/comentarios NO son coloreados
% - los numeros por fuera que pertenecen a una palabra SON coloreados
%   (buf1, por ejemplo, el "1" es coloreado cuando no lo deberias. UPS!!)
% - No incluye el signo
%
%\lstset{literate=%
%  *{0}{{{\color{violet}0}}}1
%   {1}{{{\color{violet}1}}}1
%   {2}{{{\color{violet}2}}}1
%   {3}{{{\color{violet}3}}}1
%   {4}{{{\color{violet}4}}}1
%   {5}{{{\color{violet}5}}}1
%   {6}{{{\color{violet}6}}}1
%   {7}{{{\color{violet}7}}}1
%   {8}{{{\color{violet}8}}}1
%   {9}{{{\color{violet}9}}}1
%}

% Para resaltar lineas de codigo usando \btLstHLB{range} y \btLstHLB<overlay>{range}:
% Por ejemplo,
%  \btLstHLB{3}       linea 3 resaltada
%  \btLstHLB{1-5}     lineas de la 1 a la 5 resaltadas
%  \btLstHLB<2>{3}    linea 3 resaltada solo en el slide 2
%
% \btLstHLB usa un color azul mientras que \btLstHLR usa un color rojo como fondo
\usepackage{lstlinebgrd}
\makeatletter
\newcount\bt@rangea
\newcount\bt@rangeb

\newcommand\btIfInRange[2]{%
   \global\let\bt@inrange\@secondoftwo%
   \edef\bt@rangelist{#2}%
   \foreach \range in \bt@rangelist {%
      \afterassignment\bt@getrangeb%
      \bt@rangea=0\range\relax%
      \pgfmathtruncatemacro\result{ ( #1 >= \bt@rangea) && (#1 <= \bt@rangeb) }%
      \ifnum\result=1\relax%
      \breakforeach%
      \global\let\bt@inrange\@firstoftwo%
      \fi%
   }%
   \bt@inrange%
}
\newcommand\bt@getrangeb{%
   \@ifnextchar\relax%
   {\bt@rangeb=\bt@rangea}%
   {\@getrangeb}%
}
\def\@getrangeb-#1\relax{%
   \ifx\relax#1\relax%
      \bt@rangeb=100000%   \maxdimen is too large for pgfmath
   \else%
      \bt@rangeb=#1\relax%
   \fi%
}

%%%%%%%%%%%%%%%%%%%%%%%%%%%%%%%%%%%%%%%%%%%%%%%%%%%%%%%%%%%%%%%%%%%%%%%%%%%%%%
%
% \btLstHL<overlay spec>{range list}
%
\newcommand<>{\btLstHLB}[1]{%
   \only#2{\btIfInRange{\value{lstnumber}}{#1}{\color{blue!30}}% blue
   {\def\lst@linebgrdcmd####1####2####3{}}}%
}%
\newcommand<>{\btLstHLR}[1]{%
   \only#2{\btIfInRange{\value{lstnumber}}{#1}{\color{red!30}}% red
   {\def\lst@linebgrdcmd####1####2####3{}}}%
}%
%
%
%%%%%%%%%%%%%%%%%%%%%%

\makeatother


% sin fecha
\date{}

\author[7542-9508]{Di Paola Mart\'in \\ \texttt{martinp.dipaola <at> gmail.com} }

\institute[Universidad de Buenos Aires]
{
   Facultad de Ingenier\'ia\\
   Universidad de Buenos Aires
}

% Definimos una imagen para que este en cada slide
%\pgfdeclareimage[height=0.5cm]{university-logo}{imgs/fiuba.png}
%\logo{\pgfuseimage{university-logo}}

%% Definir estas en el documento final
%%
%% \title%
%% {Programaci\'on gen\'erica y templates en C++}
%%
%% \subject{Programaci\'on gen\'erica y templates en C++}


%%%%%
~% if interactive is on %~
    \AtBeginSection{}
    \AtBeginSubsection{}
~% else %~
    % At begin of each Section do nothing; at each Subsection show the Section and Subsection names
    % If the Section doesn't have a Subsection, then YOU must to add a unnumbered Subsection like \subsection*{}
    % so the AtBeginSubsection will get triggered
    \AtBeginSection{}
    \AtBeginSubsection[\frame{\subsectionpage}]{\frame{\subsectionpage}}
~% endif %~
%
%%%%%




\title%
{Clases en C++}

\subject{Clases en C++}

\begin{document}

\begin{frame}
   \titlepage
\end{frame}

\begin{frame}{De qu\'e va esto?}
   \tableofcontents
   % You might wish to add the option [pausesections]
\end{frame}

\section{structs y clases en C++}
\subsection{Bundle}
\begin{frame}[plain,fragile]{TDAs - Clases en C}{}
   \begin{columns}[t]
      \begin{column}{.4\linewidth}
         \begin{lstlisting}[style=normal,firstnumber=1]
struct Vector {
  int *_data; /*private*/
  int _size; /*private*/
};


         \end{lstlisting}
      \end{column}
      \begin{column}{.4\linewidth}
         \begin{lstlisting}[style=normal,firstnumber=15]
void f() {
  struct Vector v;
  vector_create(&v, 5);
  vector_get(&v, 0);
  vector_destroy(&v);
}
         \end{lstlisting}
      \end{column}
   \end{columns}

   \begin{columns}[t]
      \begin{column}{.4\linewidth}
        \begin{lstlisting}[style=normal,firstnumber=5]
void vector_create(struct Vector *v, int size) {
  v->_data = malloc(size*sizeof(int));
  v->_size = size;
}
int vector_get(struct Vector *v, int pos) {
  return v->_data[pos];
}
void vector_destroy(struct Vector *v) {
  free(v->_data);
}

        \end{lstlisting}
      \end{column}
      \begin{column}{.4\linewidth}
      \end{column}
   \end{columns}
\end{frame}
\note[itemize] {
\item El est\'andar de C++ garantiza que si un \lstinline[style=normal]!struct! no tiene ningun feature de C++ (o sea, se parece a un struct de C) se lo llama "plain struct" y puede ser usado por libs de C desde C++ 
\item Por convenci\'on los nombres de las funciones del TDA deben tener como prefijo de su nombre el nombre del TDA: esto es por que en C todas las funciones terminan en el mismo espacio global y deben tener nombres \'unicos. El conflicto de nombres es un problema com\'un en proyectos grandes en C. En C++ tenemos mejores formas de resolverlos....
\item Asi tambi\'en es convenci\'on pasar como primer argumento un puntero al \lstinline[style=normal]!struct!. Veremos que en C++ hay una forma m\'as conveniente de hacer esto...
\item Y otra convenci\'on mas: los atributos que no deber\'ian ser ni le\'idos ni modificados por el usuario son marcados como privados. Dependiendo de la convenci\'on hay gente que le pone un guion bajo al principio de la variable, otros al final y otros ponen solamente un comentario. C++ nos dara herramientas para forzar esto en tiempo de compilaci\'on.
}


\begin{frame}[plain,fragile]{Keyword struct impl\'icita}
   \begin{columns}[t]
      \begin{column}{.4\linewidth}
         \begin{lstlisting}[style=normal,firstnumber=1]
struct Vector {
  int *_data; /*private*/
  int _size; /*private*/
};


         \end{lstlisting}
      \end{column}
      \begin{column}{.4\linewidth}
         \begin{lstlisting}[style=normal,firstnumber=15,linebackgroundcolor={%
                 \btLstHLB<1>{16}%
         }]
void f() {
  Vector v;
  vector_create(&v, 5);
  vector_get(&v, 0);
  vector_destroy(&v);
}
         \end{lstlisting}
      \end{column}
   \end{columns}

   \begin{columns}[t]
      \begin{column}{.4\linewidth}
        \begin{lstlisting}[style=normal,firstnumber=5]
void vector_create(Vector *v, int size) {
  v->_data = malloc(size*sizeof(int));
  v->_size = size;
}
int vector_get(Vector *v, int pos) {
  return v->_data[pos];
}
void vector_destroy(Vector *v) {
  free(v->_data);
}

        \end{lstlisting}
      \end{column}
      \begin{column}{.4\linewidth}
      \end{column}
   \end{columns}
\end{frame}
\note[itemize] {
\item En C++ no es necesario usar la keyword \lstinline[style=normal]!struct! en todos lados.
}

\begin{frame}[fragile]{Bundle: atributos + m\'etodos}
        \begin{lstlisting}[style=normal,firstnumber=1]
struct Vector {
  int *_data; /*private*/
  int _size; /*private*/

  void vector_create(Vector *v, int size) {
    v->_data = malloc(size*sizeof(int));
    v->_size = size;
  }

  int vector_get(Vector *v, int pos) {
    return v->_data[pos];
  }

  void vector_destroy(Vector *v) {
    free(v->_data);
  }
};
        \end{lstlisting}
\end{frame}
\note[itemize] {
\item Se integran las funciones y los datos del TDA en una sola unidad.
\item Los datos del TDA se lo llaman atributos y las funciones m\'etodos.
}

\begin{frame}[fragile]{this: un puntero a la instancia}
        \begin{lstlisting}[style=normal,firstnumber=1,linebackgroundcolor={%
                 \btLstHLB<1>{5,10,14}%
         }]
struct Vector {
  int *_data; /*private*/
  int _size; /*private*/

  void vector_create(int size) {
    this->_data = malloc(size*sizeof(int));
    this->_size = size;
  }

  int vector_get(int pos) {
    return this->_data[pos];
  }

  void vector_destroy() {
    free(this->_data);
  }
};
        \end{lstlisting}
\end{frame}
\note[itemize] {
\item Las funciones del TDA pasan a ser m\'etodos del TDA y reciben como par\'ametro impl\'icito un puntero a la instancia.
\item El puntero es un puntero constante a la instancia (\lstinline[style=normal]!Vector *const!) y se lo nombra con la keyword \lstinline[style=normal]!this!. En otras palabras \lstinline[style=normal]!this! es un puntero constante que apunta al objeto sobre el cual se esta invocando el m\'etodo.
}

\begin{frame}[fragile]{Invocaci\'on de m\'etodos}
   \begin{columns}[t]
      \begin{column}{.4\linewidth}
         \begin{lstlisting}[style=normal,firstnumber=14]
// En C
void f() {
  struct Vector v;
  vector_create(&v, 5);
  vector_get(&v, 0);

  v._data;

  vector_destroy(&v);
}
         \end{lstlisting}
     \end{column}
      \begin{column}{.4\linewidth}
        \begin{lstlisting}[style=normal,firstnumber=14]
// En C++
void f() {
  Vector v;
  v.vector_create(5);
  v.vector_get(0);

  v._data;

  v.vector_destroy();
}
        \end{lstlisting}
    \end{column}
\end{columns}
\end{frame}
\note[itemize] {
\item Se accede a los atributos y/o m\'etodos como en C
\item En C, la instancia sobre la que se quiere invocar un m\'etodo es pasada como par\'ametro de forma expl\'icita mientras que en C++ es impl\'icita e invisible.
}

\begin{frame}[fragile]{Reducci\'on de colisiones de nombres}
        \begin{lstlisting}[style=normal,firstnumber=1,linebackgroundcolor={%
                 \btLstHLB<1>{5,10,14}%
         }]
struct Vector {
  int *_data; /*private*/
  int _size; /*private*/

  void create(int size) { // Vector::create
    this->_data = malloc(size*sizeof(int));
    this->_size = size;
  }

  int get(int pos) { // Vector::get
    return this->_data[pos];
  }

  void destroy() { // Vector::destroy
    free(this->_data);
  }
};
        \end{lstlisting}
\end{frame}
\note[itemize] {
\item Los m\'etodos de un TDA no entran en conflicto con otros aunque se llamen iguales. El m\'etodo \lstinline[style=normal]!get! de \lstinline[style=normal]!Vector! no entra en conflicto con el m\'etodo \lstinline[style=normal]!get! de \lstinline[style=normal]!Matrix!, por ejemplo
\item En rigor un m\'etodo de un TDA se lo llama \lstinline[style=normal]!NombreTDA::NombreMetodo!, por eso \lstinline[style=normal]!Vector::get! es distinto de \lstinline[style=normal]!Matrix::get!.
\item Veremos con mas detalle el concepto de namespace en las pr\'oximas clases.
}


\subsection{Permisos de acceso}
\begin{frame}[plain,fragile]{Permisos de acceso}
        \begin{lstlisting}[style=normal,firstnumber=1,linebackgroundcolor={%
                 \btLstHLB<1>{2,6}%
                 \btLstHLB<2>{2-4}%
                 \btLstHLB<3>{6-18}%
         }]
struct Vector {
  private:
  int *data;
  int size;

  public:
  void create(int size) {
    this->data = malloc(size*sizeof(int));
    this->size = size;
  }

  int get(int pos) { 
    return this->data[pos];
  }

  void destroy() { 
    free(this->data);
  }
};
        \end{lstlisting}
\end{frame}
\note[itemize] {
\item Por default, un \lstinline[style=normal]!struct! tiene sus atributos y m\'etodos p\'ublicos. Esto significa que pueden accederse desde cualquier lado.
\item Se puede cambiar el default forzando distintos permisos.
\item \lstinline[style=normal]!private! hace que s\'olo los m\'etodos internos puedan acceder a los m\'etodos y atributos privados.
\item M\'as sobre los permisos \lstinline[style=normal]!public/protected/private! y su relaci\'on con la herencia en las pr\'oximas clases.
}

\begin{frame}[fragile]{Permisos de acceso}{}
   \begin{columns}[t]
      \begin{column}{.4\linewidth}
         \begin{lstlisting}[style=normal,firstnumber=14]
// En C
void f() {
  struct Vector v;
  vector_create(&v, 5);
  vector_get(&v, 0);

  v._data;

  vector_destroy(&v);
}
         \end{lstlisting}
     \end{column}
      \begin{column}{.4\linewidth}
        \begin{lstlisting}[style=normal,firstnumber=14,linebackgroundcolor={%
                 \btLstHLR<1>{20}%
         }]
// En C++
void f() {
  Vector v;
  v.create(5);
  v.get(0);

  v.data;

  v.destroy();
}
        \end{lstlisting}
    \end{column}
\end{columns}
\end{frame}

\subsection{Clases}
\begin{frame}[fragile]{Clases en C++}
        \begin{lstlisting}[style=normal,firstnumber=1]
struct Vector {
  int *data;  // public by default
  int size;   // public by default
};

class Vector {
  int *data;  // private by default
  int size;   // private by default
};
        \end{lstlisting}
\end{frame}
\note[itemize] {
\item Absolutamente todo lo visto con structs en C++ aplica a las clases de C++. La \'unica diferencia es que las clases tienen sus atributos y m\'etodos privados por default.
}


\begin{frame}[plain,fragile]{Unidades de compilaci\'on}{}
   \begin{columns}[t]
      \begin{column}{.4\linewidth}
        \begin{lstlisting}[style=normal,firstnumber=1]
class Vector {
  private:
  int *data;
  int size;

  public:
  void create(int size);
  int get(int pos);
  void destroy();

}; // en el archivo vector.h
        \end{lstlisting}
     \end{column}
     \begin{column}{.4\linewidth}
        \begin{lstlisting}[style=normal,firstnumber=1]
#include "vector.h"
void Vector::create(int size) {
  this->data = malloc(
  this->size = size;
}

int Vector::get(int pos) {
  return this->data[pos];
}

void Vector::destroy() {
  free(this->data);
} // en el archivo vector.cpp
        \end{lstlisting}
    \end{column}
\end{columns}
\end{frame}
\note[itemize] {
\item Hasta ahora se integr\'o en un solo lugar el c\'odigo de cada m\'etodo. Es m\'as simple pero trae problemas de performance del proceso de compilaci\'on.
\item Para evitar recompilar una y otra vez el c\'odigo de los m\'etodos se le define en un archivo .cpp separado de las declaraciones del .h
}

\section{RAII: Resource Acquisition Is Initialization}
\subsection{Constructor y destructor}
\begin{frame}[fragile]{Constructor/destructor: manejo de recursos autom\'atico}
        \begin{lstlisting}[style=normal,firstnumber=1,linebackgroundcolor={%
                 \btLstHLB<1>{5-8}%
                 \btLstHLB<2>{14-16}%
         }]
struct Vector {
  int *data;
  int size;

  Vector(int size) {  // create
    this->data = malloc(size*sizeof(int));
    this->size = size;
  }

  int get(int pos) {
    return this->data[pos];
  }

  ~Vector() { // destroy
    free(this->data);
  }
};
        \end{lstlisting}
\end{frame}
\note[itemize] {
\item El constructor es un c\'odigo que se ejecuta al momento de crear un nuevo objeto. C++ siempre llama a algun constructor al crear un nuevo objeto.
\item Todos los objetos son creados por un constructor. Si un TDA no tiene un constructor, C++ crea un constructor por default.
\item Un TDA puede tener m\'ultiples constructores (que los veremos a continuaci\'on). Sin embargo s\'olo puede haber un \'unico destructor.
\item Un destructor es un c\'odigo que se ejecuta al momento de destuirse un objeto (cuando este se va de scope o es eliminado del heap con \lstinline[style=normal]!delete!).
\item Todos los objetos tienen un destructor. Si un TDA no tiene un destructor, C++ crea un destructor por default.
}

\begin{frame}[fragile]{Reduciendo la probabilidad de errores}
\only<1>{Diferencia entre reservar memoria y construir un objeto}
\only<2>{Memoria sin inicializar}
\only<3>{Destrucci\'on autom\'atica}
   \begin{columns}[t]
      \begin{column}{.4\linewidth}
         \begin{lstlisting}[style=normal,firstnumber=29,linebackgroundcolor={%
                 \btLstHLB<1>{31}%
                 \btLstHLR<2>{33,35}%
                 \btLstHLR<3>{38}%
         }]
// En C
void g() {
  struct Vector v;

  v.data;

  vector_create(&v, 5);
  //...

}
         \end{lstlisting}
     \end{column}
      \begin{column}{.4\linewidth}
         \begin{lstlisting}[style=normal,firstnumber=29,linebackgroundcolor={%
                 \btLstHLB<1>{31}%
                 \btLstHLB<2>{31,33}%
                 \btLstHLB<3>{38}%
         }]
// En C ++
void g() {
  Vector v(5);

  v.data;


  //...

}
         \end{lstlisting}
     \end{column}
 \end{columns}
\end{frame}
\note[itemize] {
\item Con los constructores (si estan bien escritos) no se puede usar un objeto sin inicializar.
\item Con los destructores (si estan bien escritos, se usa RAII y usamos el stack) no vamos a tener leaks.
\item Los destructores se llaman autom\'aticamente cuando el objeto se va de scope.
}

\begin{frame}[fragile]{Operadores new y delete}
        \begin{lstlisting}[style=normal,firstnumber=1,linebackgroundcolor={%
                 \btLstHLB<1>{6,15}%
         }]
struct Vector {
  int *data;
  int size;

  Vector(int size) {
    this->data = new int[size]();
    this->size = size;
  }

  int get(int pos) {
    return this->data[pos];
  }

  ~Vector() {
    delete[] this->data;
  }
};
        \end{lstlisting}
\end{frame}
\note[itemize] {
\item Las funciones \lstinline[style=normal]!malloc! y \lstinline[style=normal]!free! reservan y liberan memoria pero no crean objetos (no llaman a los constructores ni los destruyen)
\item El operador \lstinline[style=normal]!new! y su contraparte \lstinline[style=normal]!delete! no s\'olo manejan la memoria del heap sino que tambi\'en llaman al respectivo constructor y destructor.
\item Para crear un array de objetos hay que usar los operadores \lstinline[style=normal]!new[]! y \lstinline[style=normal]!delete[]! y la clase a instanciar debe tener un constructor sin par\'ametros.
\item Hay una sutil diferencia sint\'actica cuando de tipos primitivos se trata, como \lstinline[style=normal]!int! o \lstinline[style=normal]!char!. La expresi\'on \lstinline[style=normal]!new int! crea un \lstinline[style=normal]!int! sin inicializar mientras que \lstinline[style=normal]!new int()! lo inicializa a cero.
}

\subsection{Manejo de errores}
\begin{frame}[plain,fragile]{Manejo de errores en C (madness)}
        \begin{lstlisting}[style=normal]
int process() {
  char *buf = malloc(sizeof(char)*20);

  FILE *f = fopen("data.txt", "rt");
  if (!f) { free(buf); return -1;}

  int s = fread(buf, sizeof(char), 20, f);

  if (s != 20) {
    fclose(f);
    free(buf);
    return -1;
  }

  fclose(f);
  free(buf);
  return 0;
}
        \end{lstlisting}
\end{frame}
\note[itemize] {
\item En C hay que chequear los valores de retorno para ver si hubo un error o no.
\item En caso de error se suele abortar la ejecuci\'on de la funci\'on actual requiriendo previamente liberar los recursos adquiridos
\item El problema esta en que es muy f\'acil equivocarse y liberar un recurso aun no adquirido u olvidarse de liberar un recurso que si lo fue.
\item No s\'olo es una cuesti\'on de leaks de memoria. Datos corruptos por archivos o sockets mal cerrados o leaks en el sistema operativo son otros factores que no se solucionan simplemente con un garbage collector ni reiniciando el programa.
}


\begin{frame}[fragile]{Aplicaci\'on del idiom RAII}
        \begin{lstlisting}[style=normal]
struct Buffer {
  Buffer(int size) {
    this->data = new char[size];
  }
  ~Buffer() {
    delete[] this->data;
  }
};

struct File {
  File(const char *name, const char *flags) {
    this->f = fopen(name, flags);
    if (!this->f) throw std::exception("fopen failed");
  }
  ~File() {
    fclose(this->f);
  }
};
        \end{lstlisting}
\end{frame}
\note[itemize] {
\item La idea es simple, si hay un recurso (memoria en el heap, un archivo, un socket) hay que encapsular el recurso en un objeto de C++ cuyo constructor lo adquiera e inicialize y cuyo destructor lo libere.
\item N\'otese como la clave esta en el dise\~no sim\'etrico del par constructor-destructor.
\item Vamos a refinar el concepto RAII en las pr\'oximas clases con el concepto de excepciones.
}


\begin{frame}[fragile]{RAII + Stack: No leaks}
        \begin{lstlisting}[style=normal]
int process() {
  Buffer buf(20);

  File f("data.txt", "rt");
  int s = f.read(buf, sizeof(char), 20, f);

  if (s != 20)
    return -1;

  return 0;
} // <-- ~File()
  //     ~Buffer()
        \end{lstlisting}
\end{frame}
\note[itemize] {
\item Tomese un minuto para reflexionar. Vea el c\'odigo y comparelo con otros c\'odigos de otras personas o incluso de usted mismo. Es la difencia entre alguien que sabe C++ de alguien que escribe c\'odigo que compila.
\item Al instanciarse los objetos RAII en el stack, sus constructores adquieren los recursos autom\'aticamente.
\item Al irse de scope cada objeto se les invoca su destructor autom\'aticamente y por ende liberan sus recursos sin necesidad de hacerlo expl\'icitamente.
\item El c\'odigo C++ se simplifica y se hace m\'as robusto a errores de programaci\'on:  RAII + Stack es uno de los conceptos claves en C++.
\item Veremos mas sobre RAII, manejo de errores y excepciones en C++ en las pr\'oximas clases.
}

\section{Constantes}
\subsection{Constantes}
\begin{frame}[fragile]{M\'etodos constantes: no modifican al objeto}
        \begin{lstlisting}[style=normal,firstnumber=1,linebackgroundcolor={%
                 \btLstHLB<1>{9-11}%
         }]
struct Vector {
  int *data;
  int size;

  void set(int pos, int val) {
      this->data[pos] = val;
  }

  int get(int pos) const {
      return this->data[pos];
  }

  /* ... */
};
        \end{lstlisting}
\end{frame}
\note[itemize] {
\item Un m\'etodo constante es un m\'etodo que no modifica el estado interno del objeto. Esto es, no cambia ning\'un atributo ni llama a ning\'un m\'etodo salvo que este sea tambi\'en constante.
\item Sirve para detectar errores en el c\'odigo en tiempo de compilaci\'on: si un m\'etodo no modifica el estado deber\'ia poderse ponerle la keyword \lstinline[style=normal]!const!; si el compilador falla es por que hay un bug en el c\'odigo y nuestra hip\'otesis de que el m\'etodo no cambiaba el estado interno del objeto es err\'onea.
}

\begin{frame}[fragile]{Objetos constantes}
        \begin{lstlisting}[style=normal,firstnumber=17]
void f() {
  Vector v(5);

  v.set(0, 1); // no const
  v.get(0); // const
}

        \end{lstlisting}
        \pause
        \begin{lstlisting}[style=normal,firstnumber=24,linebackgroundcolor={%
                 \btLstHLR<2>{27}%
         }]
void f() {
  const Vector v(5); // objeto constante

  v.set(0, 1); // no const
  v.get(0); // const
}
        \end{lstlisting}
\end{frame}

\begin{frame}[fragile]{Const como promesa}
        \begin{lstlisting}[style=normal,firstnumber=17,linebackgroundcolor={%
                 \btLstHLB<1>{23}%
                 \btLstHLR<1>{24}%
         }]
void f() {
  Vector v(5);

  g(v);
}

void g(const Vector &v) {
  v.set(0, 1); // no const
  v.get(0); // const
}
        \end{lstlisting}
\end{frame}
\note[itemize] {
\item Es comun recibir par\'ametros constantes. La funci\'on promete que no va a cambiar al objeto recibido como par\'ametro.
}

\begin{frame}[fragile]{Atributos constantes}
        \begin{lstlisting}[style=normal,firstnumber=1,linebackgroundcolor={%
                 \btLstHLB<1>{2,3}%
                 \btLstHLB<2>{2,3,10}%
                 \btLstHLB<3>{2,3,6}%
         }]
struct Vector {
  int * const data; // no confundir con int const * data;
  const int size; // equivalente a int const size;

  void set(int pos, int val) {
      this->data[pos] = val;
  }

  int get(int pos) const {
      return this->data[pos];
  }

  /* ... */
};
        \end{lstlisting}
\end{frame}
\note[itemize] {
\item Tambi\'en podemos tener atributos constantes. Estos toman un valor cuando se crean y lo mantienen durante toda la vida del objeto.
\item Peque\~na aclaraci\'on: \lstinline[style=normal]!const int! y \lstinline[style=normal]!int const! son equivalentes asi como tambi\'en \lstinline[style=normal]!const int *! y \lstinline[style=normal]!int const *!. Sin embargo es distinto \lstinline[style=normal]!int * const!. Confuso?
\item \lstinline[style=normal]!const int * p! se lee como "\lstinline[style=normal]!p! es un puntero; a \lstinline[style=normal]!int!; constante" mientras que \lstinline[style=normal]!int * const p! se lee como "\lstinline[style=normal]!p! es constante; puntero; a \lstinline[style=normal]!int!". El primero apunta a \lstinline[style=normal]!int!s constantes mientras que el segundo es el puntero quien es constante.
}

\subsection{Initialization}
\begin{frame}[fragile]{Member Initialization List}
        \begin{lstlisting}[style=normal,firstnumber=1,linebackgroundcolor={%
                 \btLstHLB<1>{5}%
                 \btLstHLB<2>{6-8}%
         }]
struct Vector {
  int *data;
  int size;

  Vector(int size) {
    // atributos ya construidos;  aca solo los re-asigno
    this->data = malloc(size*sizeof(int));
    this->size = size;
  }
        \end{lstlisting}
        \pause
        \pause
        \begin{lstlisting}[style=normal,firstnumber=1,linebackgroundcolor={%
                 \btLstHLB<3>{5-6}%
         }]
struct Vector {
  int *data;
  int size;

  Vector(int size) : data(malloc(size*sizeof(int))),
                     size(size) {

  }
        \end{lstlisting}
\end{frame}
\note[itemize] {
\item Al ejecutarse el cuerpo del constructor todos sus atributos ya estan creados.
\item Si se necesita construir alguno o todos sus atributos con par\'amentros especiales hay que usar la member initialization list.
\item Esto es \'util no s\'olo para crear objetos que no pueden cambiar una vez construidos (como los atributos \lstinline[style=normal]!const! y las referecias) sino que tambi\'en es necesario si queremos construir otros objetos con par\'ametros custom, sean nuestros atributos o nuestros ancestros (herencia).
}

\begin{frame}[fragile]{Inicializaci\'on de atributos constantes}
        \begin{lstlisting}[style=normal,firstnumber=1,linebackgroundcolor={%
                 \btLstHLR<1>{2,7}%
                 \btLstHLR<2>{3,8}%
         }]
struct Vector {
  int * const data;
  const int size;

  Vector(int size) {
    // atributos ya construidos;  aca solo los re-asigno
    this->data = malloc(size*sizeof(int));
    this->size = size;
  }
        \end{lstlisting}
        \pause
        \pause
        \begin{lstlisting}[style=normal,firstnumber=1,linebackgroundcolor={%
                 \btLstHLB<3>{2,5}%
                 \btLstHLB<4>{3,6}%
         }]
struct Vector {
  int * const data;
  const int size;

  Vector(int size) : data(malloc(size*sizeof(int))),
                     size(size) {

  }
        \end{lstlisting}
\end{frame}
\note[itemize] {
\item La member initialization list es el \'unico lugar para inicializar atributos constantes y referencias.
}

\begin{frame}[fragile]{Inicializaci\'on de atributos no-default}
    \begin{lstlisting}[style=normal,firstnumber=1,linebackgroundcolor={%
                 \btLstHLR<1>{5}%
         }]
struct DoubleVector {
  Vector fg;
  Vector bg;

  DoubleVector(int size) {
      // fg, bg??
  }
}
        \end{lstlisting}
        \pause
    \begin{lstlisting}[style=normal,firstnumber=1,linebackgroundcolor={%
                 \btLstHLB<2>{5}%
         }]
struct DoubleVector {
  Vector fg;
  Vector bg;

  DoubleVector(int size) : fg(size), bg(size) {
  }
}
        \end{lstlisting}
\end{frame}
\note[itemize] {
\item La member initialization list es el \'unico lugar para inicializar atributos que son objetos que no tienen un constructor por default o sin par\'ametros.
}

\begin{frame}[fragile]{Delegating constructors}
    \begin{lstlisting}[style=normal,firstnumber=1]
struct DoubleVector {
  DoubleVector(int size) : fg(size), bg(size) { }

  DoubleVector(int size, int val) : fg(size), bg(size) {
      for (int i = 0; i < size; ++i) {
          fg.set(i, val);
          bg.set(i, val);
      }
        \end{lstlisting}
        \pause
    \begin{lstlisting}[style=normal,firstnumber=1]
struct DoubleVector {
  DoubleVector(int size) : fg(size), bg(size) { }

  DoubleVector(int size, int val) : DoubleVector(size) {
      for (int i = 0; i < size; ++i) {
          fg.set(i, val);
          bg.set(i, val);
      }
        \end{lstlisting}
\end{frame}
\note[itemize] {
\item La member initialization list permite llamar a otro constructor para delegarle parte de la construcci\'on del objeto. Esto permite reutilizar c\'odigo entre los constructores.
}

\appendix
\section<presentation>*{\appendixname}
\subsection<presentation>*{Referencias}

\begin{frame}[allowframebreaks]
   \frametitle<presentation>{Referencias}

   \begin{thebibliography}{10}

         \beamertemplatebookbibitems
         % Start with overview books.

      \bibitem{Stroustrup}
         Bjarne Stroustrup.
         \newblock {\em The C++ Programming Language}.
         \newblock Addison Wesley, Fourth Edition.

   \end{thebibliography}
\end{frame}

\end{document}


