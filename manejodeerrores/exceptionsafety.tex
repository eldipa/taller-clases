\input{../preamble-tmp.tex}

\title%
{Exception Safety}


\subject{Exception Safety}


\begin{document}

\begin{frame}
   \titlepage
\end{frame}

~% if interactive is off %~
\begin{frame}{De qu\'e va esto?}
   \tableofcontents
   % You might wish to add the option [pausesections]
\end{frame}
~% endif %~



~% macro content() %~
\section{Exception Safety}
\subsection{}
\begin{frame}[fragile]{No s\'olo es una cuesti\'on de leaks}
   \begin{lstlisting}[style=normal]
struct Date {
   void set_day(int day) {
      this._day = day;   if (/* invalid */) throw -1;
   }

   void set_month(int month) {
      this._month = month;  if (/* invalid */) throw -1;
   }
}
try {
   Date d(30, 04);
   d.set_day(31);
   d.set_month(01);
} catch(...) {
   std::cout << d;
}
   \end{lstlisting}
El estado final del objeto \lstinline[style=normal]!d! es \dots
~% if interactive is off -%~
\pause
\lstinline[style=normal]!31/04!, no tiene sentido!!
~% endif %~
\end{frame}
\note[itemize] {
\item Los errores pueden suceder en cualquier momento. No solo hay que evitar leaks (y otros) sino que tambi\'en hay que tratar de dejar a los objetos en un estado consistente y sin corromper (donde siguen manteniendo sus invariantes).
\item Claramente la estrategia "modifico el objeto y luego chequeo" no trae mas que problemas.
}

\begin{frame}[fragile]{Exception safe weak (o basic): objetos consistentes.}
   \begin{lstlisting}[style=normal]
struct Date {
   void set_day(int day) {
      if (/* invalid */) throw -1;  this._day = day;
   }

   void set_month(int month) {
      if (/* invalid */) throw -1;  this._month = month;
   }
}
try {
   Date d(30, 01);
   d.set_day(31);
   d.set_month(02);
} catch(...) {
   std::cout << d;
}
   \end{lstlisting}
Y ahora?
~% if interactive is off -%~
\pause
imprime \lstinline[style=normal]!31/01!, fecha v\'alida pero no es la original.
~%- endif %~
\end{frame}
\note[itemize] {
\item Con una simple modifici\'on de c\'odigo se puede mejorar mucho la situaci\'on.
\item Ante un error el objeto no queda corrupto (sigue manteniendo sus invariante) aunque puede no quedar en un estado definido.
\item Cuando ante una excepci\'on un objeto no se corrompe pero no queda en el estado anterior se dice que el m\'etodo en cuesti\'on es exception safe weak.
}

\begin{frame}[fragile]{Exception safe strong: objetos inalterados.}
   \begin{lstlisting}[style=normal]
struct Date {
   void load_date(int day, int month) {
      if (/* invalid */) throw -1;
      this.set_day(day);
      this.set_month(month);
   }

   void set_day(int day) { /* ... */ }
   void set_month(int month) { /* ... */ }
}
try {
   Date d(28, 01);
   d.load_date(31, 02);
} catch(...) {
   std::cout << d;
}
   \end{lstlisting}
Ahora imprime \lstinline[style=normal]!28/01!, el objeto \alert{no cambi\'o}.
\end{frame}
\note[itemize] {
\item Cuando ante una excepci\'on un objeto no se corrompe y adem\'as queda en el estado anterior se dice que el m\'etodo en cuesti\'on es exception safe strong.
\item Los objetos deber\'ian en lo posible implementar sus m\'etodos como exception safe strong. No siempre es posible, pero en la medida que se pueda deber\'ia intentarse.
}

\begin{frame}[fragile]{Encapsulaci\'on de objetos y Exception safety}
Los setters son un peligro, no es posible garantizar una interfaz strong exception safe:
   \begin{lstlisting}[style=normal]
   public:
      void load_date(int day, int month);
      void set_day(int day);
      void set_month(int month);
   \end{lstlisting}
Pero si la interfaz esta bien dise\~nada, es m\'as f\'acil hacer garant\'ias:
   \begin{lstlisting}[style=normal]
   public:
      void load_date(int day, int month);

   private:
      void set_day(int day);
      void set_month(int month);
   \end{lstlisting}
\end{frame}
\note[itemize] {
\item El manejo de errores y las excepciones no se pueden agregar al final de un proyecto, deben dise\~narse en conjunto con el resto del objeto pues para poder garantizar los exception safe weak/strong y se requiere de un dise\~no cuidadoso de la interfaz.
\item A grandes razgos, los m\'etodos set son los causantes de muchos problemas por que pueden dejar inv\'alidos a los objetos, sobre todo si algo falla en el medio. Evitar a toda costa los set.
\item La encapsulaci\'on de un objeto no significa "poner los atributos privados y los getters y setters p\'ublicos" sino que significa "poner todo privado salvo los m\'etodos que no pueden dejar inconsistente al objeto".
\item Por ejemplo, sea la fecha \lstinline[style=normal]!Date d(30, 04)! y se quiere cambiar a \lstinline[style=normal]!28/02! (ambas fechas v\'alidas). Usando solamente los m\'etodos \lstinline[style=normal]!set_day! y \lstinline[style=normal]!set_month!, en que orden hay que invocar a esos m\'etodos para obtener la fecha requerida? F\'acil no? Y si ahora de la fecha \lstinline[style=normal]!30/04! se quiere ir a \lstinline[style=normal]!31/01!? Vale el mismo orden o lo tuviste que cambiar? \lstinline[style=normal]!:)! Los setters son malos.
}

\begin{frame}[fragile,plain]{El dise\~no de la interfaz es afectada}
   \begin{lstlisting}[style=normal,linebackgroundcolor={%
         \only<1>{\def\lst@linebgrdcmd####1####2####3{}}%
         \btLstHLB<2>{7}%
         \btLstHLB<3>{8}%
         \btLstHLB<4>{10}%
         \only<5>{\def\lst@linebgrdcmd####1####2####3{}}%
         \btLstHLR<6>{9}%
   }]
template<class T>
T Stack::pop() {
   if (count_elements == 0) {
      throw "Stack empty";
   }
   else {
      T temp;
      temp = elements[count_elements-1];
      --count_elements;
      return temp;
   }
}
   \end{lstlisting}
    \vspace{-0.5cm}
\only<1-4>{
Qu\'e puede salir mal y lanzar una excepci\'on?
    ~% if interactive is off -%~
   \begin{itemize}
      \item<2-> El constructor por default.
      \item<3-> El operador asignaci\'on (=).
      \item<4-> El constructor por copia.
   \end{itemize}
    ~%- endif %~
}
\only<5>{
Hay posibilidad de leak? Es exception safe weak o strong?
}
~% if interactive is off -%~
\only<6->{El constructor por copia nos arruina. No hay forma de poner un \lstinline[style=normal]!try!-\lstinline[style=normal]!catch! y revertir "\lstinline[style=normal]!--count_elements!".\\
   Por eso los containers del est\'andar ofrecen dos m\'etodos, \lstinline[style=normal]!void pop()! y \lstinline[style=normal]!T top()!.
}
~%- endif %~
\end{frame}
\note[itemize] {
\item Los containers de C++ son exception safe strong.
\item Este ejemplo explica por que el \lstinline[style=normal]!std::stack! de C++ tiene un \lstinline[style=normal]!void pop()! a diferencia de otros lenguages donde el \lstinline[style=normal]!pop()! retorna el objeto sacado.
}

\begin{frame}[fragile]{Exception Safety - Resumen}
   \begin{itemize}
      \item Tratar de dejar los objetos inalterados (Exception safe strong)
      \item No poner setters. Es muy f\'acil equivocarse y dejer objetos inconsistentes.
   \end{itemize}
\end{frame}

~% endmacro %~

~{ content() }~

\end{document}


