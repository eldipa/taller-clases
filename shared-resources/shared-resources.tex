
\pdfminorversion=4

~% if handout is on %~
    \documentclass[professionalfonts,handout]{beamer}
    \setbeameroption{show notes}
    \setbeamertemplate{note page}{\insertnote}
    \setbeamercolor{background canvas}{bg=white}
~% else %~
    ~% if interactive is on %~
        \documentclass[professionalfonts,handout]{beamer}
    ~% else %~
        \documentclass[professionalfonts]{beamer}
    ~% endif %~
~% endif %~



% Template original the Beamer: Copyright 2004 by Till Tantau <tantau@users.sourceforge.net>.


\mode<presentation>
{
  \usetheme{metropolis}
  % or ...

  %\setbeamercovered{transparent} %hace que lo que esta covered se muestre transparente
}

\setbeamertemplate{navigation symbols}{}%remove navigation symbols

\usepackage[english]{babel}

\usepackage[latin1]{inputenc}

\usepackage{times}
\usepackage[T1]{fontenc}

% para poder escribir codigo fuente en las diapositivas
\usepackage{listings}

% para graficar diagramas simples y arboles de jerarquias
\usepackage{tikz}
\usepackage{tikz-qtree}
\usetikzlibrary{matrix,positioning}
\usetikzlibrary{shapes,arrows,chains,calc,decorations.pathmorphing,decorations.shapes}

\definecolor{myLightGray}{RGB}{191,191,191}
\definecolor{myGray}{RGB}{160,160,160}
\definecolor{myDarkGray}{RGB}{144,144,144}
\definecolor{myDarkRed}{RGB}{167,114,115}
\definecolor{myRed}{RGB}{255,58,70}
\definecolor{myGreen}{RGB}{0,255,71}

\usepackage{xcolor}
\definecolor{darkgreen}{rgb}{0,.5,0}
\definecolor{darkblue}{rgb}{0,0,.7}
\lstdefinestyle{normal}{language=C++,       % lenguaje C++
   numbers=left,                            % enumerar las lineas
   keywordstyle=\color{darkblue}\textbf,    % color de las keywords
   stringstyle=\color{red},                 % color de los strings
   commentstyle=\color{darkgreen},          % color de los comentarios
   basicstyle=\color{black}\ttfamily\footnotesize\bfseries,     % color del texto en general
   morecomment=[l][\color{magenta}]{\#},    % coloreamos las intrucciones del precompilador (todo lo que empieza con #)
   ndkeywords={NULL,nullptr,siz,zer,mov,add,noexcept},               % definimos una nuevas keywords como NULL y nullptr
   ndkeywordstyle=\color{violet},           % y las nuevas keywords tendran este color
   frame=simple,                            % simple, sin ningun marco o frame alrededor del codigo
   basewidth={0.55em,0.55em}                % tamano de las letras/lineas. usado para reducir el whitespace entre estas
}

\lstdefinestyle{normal33}{language=C++,       % lenguaje C++
   numbers=left,                            % enumerar las lineas
   keywordstyle=\color{darkblue}\textbf,    % color de las keywords
   stringstyle=\color{red},                 % color de los strings
   commentstyle=\color{darkgreen},          % color de los comentarios
   basicstyle=\color{black}\ttfamily\footnotesize\bfseries,     % color del texto en general
   morecomment=[l][\color{magenta}]{\#},    % coloreamos las intrucciones del precompilador (todo lo que empieza con #)
   ndkeywords={NULL,nullptr},               % definimos una nuevas keywords como NULL y nullptr
   ndkeywordstyle=\color{violet},           % y las nuevas keywords tendran este color
   frame=simple,                            % simple, sin ningun marco o frame alrededor del codigo
   basewidth={0.55em,0.55em}                % tamano de las letras/lineas. usado para reducir el whitespace entre estas
}
%mismo estilo, sin numeros
\lstdefinestyle{normalnonumbers}{language=C++,
   keywordstyle=\color{darkblue}\textbf,
   stringstyle=\color{red},
   commentstyle=\color{darkgreen},
   basicstyle=\color{black}\ttfamily\footnotesize\bfseries,
   morecomment=[l][\color{magenta}]{\#},
   ndkeywords={NULL,nullptr,move,add},
   ndkeywordstyle=\color{violet},
   frame=simple,
   basewidth={0.55em,0.55em}
}


% mismo estilo pero todos los colores estan mas debiles como si se volvieran transparentes
% usado para poder resaltar el codigo.
\lstdefinestyle{dimmided}{language=C++,
   keywordstyle=\color{darkblue!30}\textbf,
   stringstyle=\color{red!30},
   commentstyle=\color{darkgreen!30},
   basicstyle=\color{black!30}\ttfamily\footnotesize\bfseries,
   morecomment=[l][\color{magenta!30}]{\#},
   ndkeywords={NULL,nullptr,siz,zer,mov,add},
   ndkeywordstyle=\color{violet!30},
   moredelim=**[is][\only<+->{\color{black}\lstset{style=normal}}]{@}{@}, % definimos que las lineas entre arrobas (@) tendran el estilo normal (con los colores a toda intensidad)
   frame=simple,
   basewidth={0.55em,0.55em}
}
\lstdefinestyle{dimmided42}{language=C++,
   keywordstyle=\color{darkblue!30}\textbf,
   stringstyle=\color{red!30},
   commentstyle=\color{darkgreen!30},
   basicstyle=\color{black!30}\ttfamily\footnotesize\bfseries,
   morecomment=[l][\color{magenta!30}]{\#},
   ndkeywords={NULL,nullptr,siz,zer,mov,add},
   ndkeywordstyle=\color{violet!30},
   moredelim=**[is][{\color{black}\lstset{style=normal}}]{@}{@}, % definimos que las lineas entre arrobas (@) tendran el estilo normal (con los colores a toda intensidad)
   frame=simple,
   basewidth={0.55em,0.55em}
}

\lstdefinelanguage{json}{
    literate=
     *{:}{{{\color{violet}{:}}}}{1}
      {,}{{{\color{violet}{,}}}}{1}
      {\{}{{{\color{darkgreen}{\{}}}}{1}
      {\}}{{{\color{darkgreen}{\}}}}}{1}
      {[}{{{\color{darkgreen}{[}}}}{1}
      {]}{{{\color{darkgreen}{]}}}}{1},
}


\lstdefinestyle{normaljson}{language=json,  % lenguaje json
   numbers=left,                            % enumerar las lineas
   keywordstyle=\color{darkblue}\textbf,    % color de las keywords
   stringstyle=\color{red},                 % color de los strings
   commentstyle=\color{red},                % color de los comentarios
   basicstyle=\color{black}\ttfamily\footnotesize\bfseries,     % color del texto en general
   morecomment=[l][\color{magenta}]{\#},    % coloreamos las intrucciones del precompilador (todo lo que empieza con #)
   ndkeywords={NULL,nullptr},               % definimos una nuevas keywords como NULL y nullptr
   ndkeywordstyle=\color{violet},           % y las nuevas keywords tendran este color
   frame=simple,                            % simple, sin ningun marco o frame alrededor del codigo
   basewidth={0.55em,0.55em}                % tamano de las letras/lineas. usado para reducir el whitespace entre estas
}

\lstdefinelanguage{http}{
    morekeywords={
        GET,
        PUT,
        POST,
        DELETE
    }
}

\lstdefinestyle{normalhttp}{language=http,  % lenguaje HTTP
   numbers=left,                            % enumerar las lineas
   keywordstyle=\color{darkblue}\textbf,    % color de las keywords
   stringstyle=\color{red},                 % color de los strings
   commentstyle=\color{red},                % color de los comentarios
   basicstyle=\color{black}\ttfamily\footnotesize\bfseries,     % color del texto en general
   morecomment=[l][\color{magenta}]{\#},    % coloreamos las intrucciones del precompilador (todo lo que empieza con #)
   ndkeywords={NULL,nullptr},               % definimos una nuevas keywords como NULL y nullptr
   ndkeywordstyle=\color{violet},           % y las nuevas keywords tendran este color
   frame=simple,                            % simple, sin ningun marco o frame alrededor del codigo
   basewidth={0.55em,0.55em}                % tamano de las letras/lineas. usado para reducir el whitespace entre estas
}

% Pone las constantes numericas con color (violeta en este caso):
% - los numeros que estan dentro de un string/comentarios NO son coloreados
% - los numeros por fuera que pertenecen a una palabra SON coloreados
%   (buf1, por ejemplo, el "1" es coloreado cuando no lo deberias. UPS!!)
% - No incluye el signo
%
%\lstset{literate=%
%  *{0}{{{\color{violet}0}}}1
%   {1}{{{\color{violet}1}}}1
%   {2}{{{\color{violet}2}}}1
%   {3}{{{\color{violet}3}}}1
%   {4}{{{\color{violet}4}}}1
%   {5}{{{\color{violet}5}}}1
%   {6}{{{\color{violet}6}}}1
%   {7}{{{\color{violet}7}}}1
%   {8}{{{\color{violet}8}}}1
%   {9}{{{\color{violet}9}}}1
%}

% Para resaltar lineas de codigo usando \btLstHLB{range} y \btLstHLB<overlay>{range}:
% Por ejemplo,
%  \btLstHLB{3}       linea 3 resaltada
%  \btLstHLB{1-5}     lineas de la 1 a la 5 resaltadas
%  \btLstHLB<2>{3}    linea 3 resaltada solo en el slide 2
%
% \btLstHLB usa un color azul mientras que \btLstHLR usa un color rojo como fondo
\usepackage{lstlinebgrd}
\makeatletter
\newcount\bt@rangea
\newcount\bt@rangeb

\newcommand\btIfInRange[2]{%
   \global\let\bt@inrange\@secondoftwo%
   \edef\bt@rangelist{#2}%
   \foreach \range in \bt@rangelist {%
      \afterassignment\bt@getrangeb%
      \bt@rangea=0\range\relax%
      \pgfmathtruncatemacro\result{ ( #1 >= \bt@rangea) && (#1 <= \bt@rangeb) }%
      \ifnum\result=1\relax%
      \breakforeach%
      \global\let\bt@inrange\@firstoftwo%
      \fi%
   }%
   \bt@inrange%
}
\newcommand\bt@getrangeb{%
   \@ifnextchar\relax%
   {\bt@rangeb=\bt@rangea}%
   {\@getrangeb}%
}
\def\@getrangeb-#1\relax{%
   \ifx\relax#1\relax%
      \bt@rangeb=100000%   \maxdimen is too large for pgfmath
   \else%
      \bt@rangeb=#1\relax%
   \fi%
}

%%%%%%%%%%%%%%%%%%%%%%%%%%%%%%%%%%%%%%%%%%%%%%%%%%%%%%%%%%%%%%%%%%%%%%%%%%%%%%
%
% \btLstHL<overlay spec>{range list}
%
\newcommand<>{\btLstHLB}[1]{%
   \only#2{\btIfInRange{\value{lstnumber}}{#1}{\color{blue!30}}% blue
   {\def\lst@linebgrdcmd####1####2####3{}}}%
}%
\newcommand<>{\btLstHLR}[1]{%
   \only#2{\btIfInRange{\value{lstnumber}}{#1}{\color{red!30}}% red
   {\def\lst@linebgrdcmd####1####2####3{}}}%
}%
%
%
%%%%%%%%%%%%%%%%%%%%%%

\makeatother


% sin fecha
\date{}

\author[7542-9508]{Di Paola Mart\'in \\ \texttt{martinp.dipaola <at> gmail.com} }

\institute[Universidad de Buenos Aires]
{
   Facultad de Ingenier\'ia\\
   Universidad de Buenos Aires
}

% Definimos una imagen para que este en cada slide
%\pgfdeclareimage[height=0.5cm]{university-logo}{imgs/fiuba.png}
%\logo{\pgfuseimage{university-logo}}

%% Definir estas en el documento final
%%
%% \title%
%% {Programaci\'on gen\'erica y templates en C++}
%%
%% \subject{Programaci\'on gen\'erica y templates en C++}


%%%%%
~% if interactive is on %~
    \AtBeginSection{}
    \AtBeginSubsection{}
~% else %~
    % At begin of each Section do nothing; at each Subsection show the Section and Subsection names
    % If the Section doesn't have a Subsection, then YOU must to add a unnumbered Subsection like \subsection*{}
    % so the AtBeginSubsection will get triggered
    \AtBeginSection{}
    \AtBeginSubsection[\frame{\subsectionpage}]{\frame{\subsectionpage}}
~% endif %~
%
%%%%%




\title%
{Multithreading y recursos compartidos}

\subject{Multithreading y recursos compartidos}

\begin{document}

\begin{frame}
   \titlepage
\end{frame}

\begin{frame}[fragile,label=RM]{Multithreading}
         \begin{lstlisting}[style=normal]
int counter = 0;

void inc() {
    ++counter;
}

int main(int argc, char* argv[]) {
    std::thread t1 {inc};
    std::thread t2 {inc};

    t1.join(); t2.join();
    return counter;
}
         \end{lstlisting}
\end{frame}
\note[itemize] {
\item En C++11 podemos ejecutar una funci\'on en su propio hilo con \lstinline[style=normal]!std::thread!
\item Luego, debemos esperar a que los hilos terminen sincronizando las ejecuciones con \lstinline[style=normal]!join!
}



\begin{frame}[fragile]{Instrucciones no at\'omicas}
   \begin{columns}
      \begin{column}{.5\linewidth}
         \begin{lstlisting}[style=normal]
void inc() {
    ++counter;
}
         \end{lstlisting}
      \end{column}
      \begin{column}{.45\linewidth}
         \begin{lstlisting}[style=normal]
mov eax <-- counter
add eax <-- 1
mov counter <-- eax
         \end{lstlisting}
      \end{column}
   \end{columns}
\end{frame}
\note[itemize] {
\item En general las instrucciones no son intrucciones at\'omicas. La intrucci\'on m\'as simple en C++ \lstinline[style=normal]!++counter! requiere la ejecuci\'on de tres instrucciones del microprocesador.
\item Por supuesto, los detalles de que instrucciones y cuantas son ejecutadas por el microprocesador dependen de la arquitectura de este.
}


\begin{frame}[fragile]{Acceso concurrente: caso fel\'iz}
   \begin{columns}
      \begin{column}{.5\linewidth}
          Thread 1
         \begin{lstlisting}[style=dimmided,firstnumber=0]
         @
mov eax <-- counter@@
add eax <-- 1@@
mov counter <-- eax@
---
---
---
         \end{lstlisting}
      \end{column}
      \begin{column}{.45\linewidth}
          Thread 2
         \begin{lstlisting}[style=dimmided,firstnumber=0]

---
---
---@
mov eax <-- counter@@
add eax <-- 1@@
mov counter <-- eax@
         \end{lstlisting}
      \end{column}
   \end{columns}
    \vspace{1em}
   \begin{columns}
      \begin{column}{.5\linewidth}
          Registros del Thread 1\\
\only<1>{eax = 0}
\only<2->{eax = 1}
      \end{column}
      \begin{column}{.45\linewidth}
          Registros del Thread 2\\
\only<1-3>{eax = }
\only<4>{eax = 1}
\only<5->{eax = 2}
      \end{column}
   \end{columns}
    \vspace{1.4em}
    \centering
   \begin{columns}
      \begin{column}{.5\linewidth}
          Data segment (Threads 1 y 2)\\
\only<1-2>{counter = 0}
\only<3-5>{counter = 1}
\only<6>{counter = 2}
      \end{column}
   \end{columns}
\end{frame}
\note[itemize] {
\item Los hilos comparten el heap y el data segment.
\item Por cada hilo tiene su propio stack y su propio set de registros.
\item En el caso fel\'iz las instrucciones que mutan una variable compartida no se interfieren entre s\'i y el resultado final es el esperado: \lstinline[style=normal]!counter == 2!
}

\begin{frame}[fragile]{Acceso concurrente: race condition}
   \begin{columns}
      \begin{column}{.5\linewidth}
          Thread 1
         \begin{lstlisting}[style=dimmided,firstnumber=0]
         @
mov eax <-- counter@@
add eax <-- 1@
---
---
---
mov counter <-- eax
         \end{lstlisting}
      \end{column}
      \begin{column}{.45\linewidth}
          Thread 2
         \begin{lstlisting}[style=dimmided,firstnumber=0]

---
---@
mov eax <-- counter@@
add eax <-- 1@@
mov counter <-- eax@
---
         \end{lstlisting}
      \end{column}
   \end{columns}
    \vspace{1em}
   \begin{columns}
      \begin{column}{.5\linewidth}
          Registros del Thread 1\\
\only<1>{eax = 0}
\only<2->{eax = 1}
      \end{column}
      \begin{column}{.45\linewidth}
          Registros del Thread 2\\
\only<1-2>{eax = }
\only<3>{eax = 0}
\only<4->{eax = 1}
      \end{column}
   \end{columns}
    \vspace{1.4em}
    \centering
   \begin{columns}
      \begin{column}{.5\linewidth}
          Data segment (Threads 1 y 2)\\
\only<1-4>{counter = 0}
\only<5>{counter = 1}
%\only<6>{counter = 1}
      \end{column}
   \end{columns}
\end{frame}
\note[itemize] {
\item Pero el Sistema Operativo puede decidir detener un hilo arbitrariamente y comenzar a ejecutar otro.
\item Cuando se accede a un recurso compartido, mutable, para operaciones de lectura y escritura, se abre la posibilidad de que la modificaci\'on se realize de forma incompleta. 
}

\begin{frame}[fragile]{Acceso concurrente: race condition}
   \begin{columns}
      \begin{column}{.5\linewidth}
          Thread 1
         \begin{lstlisting}[style=dimmided42,firstnumber=0]
         @
mov eax <-- counter
add eax <-- 1@
---
---
---@
mov counter <-- eax@
         \end{lstlisting}
      \end{column}
      \begin{column}{.45\linewidth}
          Thread 2
         \begin{lstlisting}[style=dimmided42,firstnumber=0]

---
---@
mov eax <-- counter
add eax <-- 1
mov counter <-- eax@
---
         \end{lstlisting}
      \end{column}
   \end{columns}
    \vspace{1em}
   \begin{columns}
      \begin{column}{.5\linewidth}
          Registros del Thread 1\\
eax = 1
      \end{column}
      \begin{column}{.45\linewidth}
          Registros del Thread 2\\
eax = 1
      \end{column}
   \end{columns}
    \vspace{1.4em}
    \centering
   \begin{columns}
      \begin{column}{.5\linewidth}
          Data segment (Threads 1 y 2)\\
counter = 1
      \end{column}
   \end{columns}
\end{frame}
\note[itemize] {
\item Una \alert{race condition} se debe al acceso no-at\'omico de lecto/escritura de un recurso compartido.
\item Si el recurso compartido es inmutable o solamente se accede a \'el para operaciones de lectura, no existe la posibilidad de tal error.
\item En el presente caso el hilo 1 no pudo completar su escritura dando resultados incorrectos. Esto se lo denomina race condition.
\item Aunque es comun ver el t\'ermino \alert{race condition} en programacion multithreading, el problema pueda darse en otros contextos como en la programaci\'on multiprocess o en la programaci\'on orientada a eventos (donde los handlers no sean at\'omicos).
}

\begin{frame}[fragile,label=RM]{Sincronizaci\'on: mutual exclusion}
         \begin{lstlisting}[style=normal]
int counter = 0;
std::mutex m;

void inc() {
    m.lock();
    ++counter;
    m.unlock();
}
         \end{lstlisting}
\end{frame}
\note[itemize] {
\item Para evitar la \alert{race condition} debemos hacer que los hilos se coordinen entre s\'i para evitar que accedan al objeto compartido a la vez.
\item Existen varias estrategias de sincronizaci\'on: sem\'aforos, colas, sockets (estos \'ultimos tambi\'en usados para comunicaci\'on)
\item Un mutex es un objeto que nos permitir\'a forzar la ejecuci\'on de un c\'odigo de forma exclusiva por un hilo a la vez. En C++ \lstinline[style=normal]!std::mutex!
}

\begin{frame}[fragile]{Acceso at\'omico}
   \begin{columns}
      \begin{column}{.5\linewidth}
          Thread 1
         \begin{lstlisting}[style=dimmided,firstnumber=0]
         @
siz m@@
mov eax <-- counter@@
add eax <-- 1@
---
mov counter <-- eax
zer m
---
         \end{lstlisting}
      \end{column}
      \begin{column}{.45\linewidth}
          Thread 2
         \begin{lstlisting}[style=dimmided,firstnumber=0]

---
---
---@
siz m@
---
---
siz m
         \end{lstlisting}
      \end{column}
   \end{columns}
    \vspace{0.8em}
   \begin{columns}
      \begin{column}{.5\linewidth}
          Registros del Thread 1\\
\only<1>{eax = }
\only<2>{eax = 0}
\only<3->{eax = 1}
      \end{column}
      \begin{column}{.45\linewidth}
          Registros del Thread 2\\
\only<1->{eax = }
      \end{column}
   \end{columns}
    \vspace{1.4em}
    \centering
   \begin{columns}
      \begin{column}{.5\linewidth}
          Data segment (Threads 1 y 2)\\
          \only<1->{counter = 0  \hspace{0.6em}  mutex = 1}
      \end{column}
   \end{columns}
\end{frame}
\note[itemize] {
\item El uso del mutex puede verse como una variable booleana. La instrucci\'on \lstinline[style=normal]!siz! setea a 1 si la variable es 0 de forma at\'omica mientras que \lstinline[style=normal]!zer! setea la variable a 0 de forma at\'omica. 
\item Cuando el hilo 2 trata de tomar or adquirir el mutex (en C++ \lstinline[style=normal]!m.lock()!), la instrucci\'on \lstinline[style=normal]!siz! falla (el variable no es 0) y el hilo deja de ejecutarse.
}

\begin{frame}[fragile]{Acceso at\'omico}
   \begin{columns}
      \begin{column}{.5\linewidth}
          Thread 1
         \begin{lstlisting}[style=dimmided42,firstnumber=0]
         @
siz m
mov eax <-- counter
add eax <-- 1@
---@
mov counter <-- eax
zer m@
---
         \end{lstlisting}
      \end{column}
      \begin{column}{.45\linewidth}
          Thread 2
         \begin{lstlisting}[style=dimmided42,firstnumber=0]

---
---
---@
siz m@
---
---
siz m
         \end{lstlisting}
      \end{column}
   \end{columns}
    \vspace{0.8em}
   \begin{columns}
      \begin{column}{.5\linewidth}
          Registros del Thread 1\\
eax = 1
      \end{column}
      \begin{column}{.45\linewidth}
          Registros del Thread 2\\
eax = 
      \end{column}
   \end{columns}
    \vspace{1.4em}
    \centering
   \begin{columns}
      \begin{column}{.5\linewidth}
          Data segment (Threads 1 y 2)\\
          counter = 1  \hspace{0.6em}  mutex = 0
      \end{column}
   \end{columns}
\end{frame}
\note[itemize] {
\item Efectivamente el uso de un mutex hace que un solo un hilo a la vez puedan acceder al objeto compartido y modificarlo.
\item Por supuesto, una vez accedido al recurso, el hilo debe liberar el mutex, de otro modo el recurso queda inaccesible para el resto de los hilos, potencialmente llevando a una situaci\'on de \alert{dead lock}.
}

\begin{frame}[fragile]{Protecci\'on de los recursos: monitor}
   \begin{columns}
      \begin{column}{.5\linewidth}
         \begin{lstlisting}[style=normal33]
class ProtectedCounter {
  int counter;
  std::mutex m;

  public:
  void inc() {
      m.lock();
      ++counter;
      m.unlock;
  }
};
         \end{lstlisting}
      \end{column}
      \begin{column}{.45\linewidth}
         \begin{lstlisting}[style=normal33]
ProtectedCounter counter;

void inc() {
    counter.inc();
}
         \end{lstlisting}
      \end{column}
   \end{columns}
\end{frame}
\note[itemize] {
\item El recurso compartido y el mutex se usan en combinaci\'on.
\item Una buena pr\'actica es la de encapsular tanto el recurso como su mutex en un objeto que sabe como protegerse a si mismo.
\item Este patr\'on es llamado Monitor y es otra forma de sincronizaci\'on que algunos lenguajes ofrecen directamente.
}

\begin{frame}[fragile]{Proteger es m\'as que usar mutexs}
   \begin{columns}
      \begin{column}{.5\linewidth}
         \begin{lstlisting}[style=normal33]
class ProtectedList {
  std::list<int> list;
  std::mutex m;

  public:
  bool has(int x) {
      m.lock();
      bool b = list.has(x);
      m.unlock;
      return b;
  }

  void add(int x) {
      m.lock();
      list.add(x);
      m.unlock;
  }
};
         \end{lstlisting}
      \end{column}
      \begin{column}{.45\linewidth}
         \begin{lstlisting}[style=normal33]
ProtectedList list;

void add_uniq(int x) {
    if (not list.has(x)) {
        list.add(x);
    }
}
         \end{lstlisting}
      \end{column}
   \end{columns}
\end{frame}
\note[itemize] {
\item Pero usar mutex/monitor no es garant\'ia: no es solo una cuesti\'on de proteger cada m\'etodo del recuso con un mutex.
\item En este caso hay una race condition: un hilo puede preguntar si un elemento est\'a en la list y si no agregarlo. Pero justo antes de llamar a \lstinline[style=normal33]!list.add! otro hilo ejecuta el mismo fragmento de c\'odigo, preguntando y agregando el mismo valor. Luego, el hilo original termina agregando un valor que ya est\'a en la lista.
\item Esto se debe a que las operaciones \lstinline[style=normal33]!list.has! y \lstinline[style=normal33]!list.add! son dependientes y deben ejecutarse de forma at\'omica. Esto es lo que se conoce como una \alert{critical section}.
}

\begin{frame}[fragile]{M\'etodos de un monitor: critical sections}
   \begin{columns}
      \begin{column}{.5\linewidth}
         \begin{lstlisting}[style=normal33]
class ProtectedList {
  std::list<int> list;
  std::mutex m;

  public:
  void add_if_hasnt(int x) {
      m.lock();
      if (not list.has(x))
          list.add(x);
      m.unlock;
  }
};
         \end{lstlisting}
      \end{column}
      \begin{column}{.45\linewidth}
         \begin{lstlisting}[style=normal33]
ProtectedList list;

void add_uniq(int x) {
    list.add_if_hasnt(x);
}
         \end{lstlisting}
      \end{column}
   \end{columns}
\end{frame}
\note[itemize] {
\item Una buena implementaci\'on de un monitor no solo encapsula el recurso y el mutex sino que adem\'as ofrece un m\'etodo por cada \alert{critical section}.
\item La intefaz p\'ublica de un monitor esta moldeada por las \alert{critical sections}.
}


\appendix
\section<presentation>*{\appendixname}
\subsection<presentation>*{Referencias}

\begin{frame}[allowframebreaks]
   \frametitle<presentation>{Referencias}

   \begin{thebibliography}{10}

         \beamertemplatebookbibitems
         % Start with overview books.

      \bibitem{Stroustrup}
         Bjarne Stroustrup.
         \newblock {\em The C++ Programming Language}.
         \newblock Addison Wesley, Fourth Edition.

      \bibitem{Tanenbaum}
         Andrew S. Tanenbaum.
         \newblock {\em Modern Operating System}.
         \newblock Prentice Hall, Second Edition.

      \bibitem{Ari}
         Mordechai Ben-Ari.
         \newblock {\em Principles of Concurrent and Distributed Programming}
         \newblock Addison Wesley, Second Edition.


   \end{thebibliography}
\end{frame}

\end{document}


