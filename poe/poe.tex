\pdfminorversion=4

~% if handout is on %~
    \documentclass[professionalfonts,handout]{beamer}
    \setbeameroption{show notes}
    \setbeamertemplate{note page}{\insertnote}
    \setbeamercolor{background canvas}{bg=white}
~% else %~
    ~% if interactive is on %~
        \documentclass[professionalfonts,handout]{beamer}
    ~% else %~
        \documentclass[professionalfonts]{beamer}
    ~% endif %~
~% endif %~



% Template original the Beamer: Copyright 2004 by Till Tantau <tantau@users.sourceforge.net>.


\mode<presentation>
{
  \usetheme{metropolis}
  % or ...

  %\setbeamercovered{transparent} %hace que lo que esta covered se muestre transparente
}

\setbeamertemplate{navigation symbols}{}%remove navigation symbols

\usepackage[english]{babel}

\usepackage[latin1]{inputenc}

\usepackage{times}
\usepackage[T1]{fontenc}

% para poder escribir codigo fuente en las diapositivas
\usepackage{listings}

% para graficar diagramas simples y arboles de jerarquias
\usepackage{tikz}
\usepackage{tikz-qtree}
\usetikzlibrary{matrix,positioning}
\usetikzlibrary{shapes,arrows,chains,calc,decorations.pathmorphing,decorations.shapes}

\definecolor{myLightGray}{RGB}{191,191,191}
\definecolor{myGray}{RGB}{160,160,160}
\definecolor{myDarkGray}{RGB}{144,144,144}
\definecolor{myDarkRed}{RGB}{167,114,115}
\definecolor{myRed}{RGB}{255,58,70}
\definecolor{myGreen}{RGB}{0,255,71}

\usepackage{xcolor}
\definecolor{darkgreen}{rgb}{0,.5,0}
\definecolor{darkblue}{rgb}{0,0,.7}
\lstdefinestyle{normal}{language=C++,       % lenguaje C++
   numbers=left,                            % enumerar las lineas
   keywordstyle=\color{darkblue}\textbf,    % color de las keywords
   stringstyle=\color{red},                 % color de los strings
   commentstyle=\color{darkgreen},          % color de los comentarios
   basicstyle=\color{black}\ttfamily\footnotesize\bfseries,     % color del texto en general
   morecomment=[l][\color{magenta}]{\#},    % coloreamos las intrucciones del precompilador (todo lo que empieza con #)
   ndkeywords={NULL,nullptr,siz,zer,mov,add,noexcept},               % definimos una nuevas keywords como NULL y nullptr
   ndkeywordstyle=\color{violet},           % y las nuevas keywords tendran este color
   frame=simple,                            % simple, sin ningun marco o frame alrededor del codigo
   basewidth={0.55em,0.55em}                % tamano de las letras/lineas. usado para reducir el whitespace entre estas
}

\lstdefinestyle{normal33}{language=C++,       % lenguaje C++
   numbers=left,                            % enumerar las lineas
   keywordstyle=\color{darkblue}\textbf,    % color de las keywords
   stringstyle=\color{red},                 % color de los strings
   commentstyle=\color{darkgreen},          % color de los comentarios
   basicstyle=\color{black}\ttfamily\footnotesize\bfseries,     % color del texto en general
   morecomment=[l][\color{magenta}]{\#},    % coloreamos las intrucciones del precompilador (todo lo que empieza con #)
   ndkeywords={NULL,nullptr},               % definimos una nuevas keywords como NULL y nullptr
   ndkeywordstyle=\color{violet},           % y las nuevas keywords tendran este color
   frame=simple,                            % simple, sin ningun marco o frame alrededor del codigo
   basewidth={0.55em,0.55em}                % tamano de las letras/lineas. usado para reducir el whitespace entre estas
}
%mismo estilo, sin numeros
\lstdefinestyle{normalnonumbers}{language=C++,
   keywordstyle=\color{darkblue}\textbf,
   stringstyle=\color{red},
   commentstyle=\color{darkgreen},
   basicstyle=\color{black}\ttfamily\footnotesize\bfseries,
   morecomment=[l][\color{magenta}]{\#},
   ndkeywords={NULL,nullptr,move,add},
   ndkeywordstyle=\color{violet},
   frame=simple,
   basewidth={0.55em,0.55em}
}


% mismo estilo pero todos los colores estan mas debiles como si se volvieran transparentes
% usado para poder resaltar el codigo.
\lstdefinestyle{dimmided}{language=C++,
   keywordstyle=\color{darkblue!30}\textbf,
   stringstyle=\color{red!30},
   commentstyle=\color{darkgreen!30},
   basicstyle=\color{black!30}\ttfamily\footnotesize\bfseries,
   morecomment=[l][\color{magenta!30}]{\#},
   ndkeywords={NULL,nullptr,siz,zer,mov,add},
   ndkeywordstyle=\color{violet!30},
   moredelim=**[is][\only<+->{\color{black}\lstset{style=normal}}]{@}{@}, % definimos que las lineas entre arrobas (@) tendran el estilo normal (con los colores a toda intensidad)
   frame=simple,
   basewidth={0.55em,0.55em}
}
\lstdefinestyle{dimmided42}{language=C++,
   keywordstyle=\color{darkblue!30}\textbf,
   stringstyle=\color{red!30},
   commentstyle=\color{darkgreen!30},
   basicstyle=\color{black!30}\ttfamily\footnotesize\bfseries,
   morecomment=[l][\color{magenta!30}]{\#},
   ndkeywords={NULL,nullptr,siz,zer,mov,add},
   ndkeywordstyle=\color{violet!30},
   moredelim=**[is][{\color{black}\lstset{style=normal}}]{@}{@}, % definimos que las lineas entre arrobas (@) tendran el estilo normal (con los colores a toda intensidad)
   frame=simple,
   basewidth={0.55em,0.55em}
}

\lstdefinelanguage{json}{
    literate=
     *{:}{{{\color{violet}{:}}}}{1}
      {,}{{{\color{violet}{,}}}}{1}
      {\{}{{{\color{darkgreen}{\{}}}}{1}
      {\}}{{{\color{darkgreen}{\}}}}}{1}
      {[}{{{\color{darkgreen}{[}}}}{1}
      {]}{{{\color{darkgreen}{]}}}}{1},
}


\lstdefinestyle{normaljson}{language=json,  % lenguaje json
   numbers=left,                            % enumerar las lineas
   keywordstyle=\color{darkblue}\textbf,    % color de las keywords
   stringstyle=\color{red},                 % color de los strings
   commentstyle=\color{red},                % color de los comentarios
   basicstyle=\color{black}\ttfamily\footnotesize\bfseries,     % color del texto en general
   morecomment=[l][\color{magenta}]{\#},    % coloreamos las intrucciones del precompilador (todo lo que empieza con #)
   ndkeywords={NULL,nullptr},               % definimos una nuevas keywords como NULL y nullptr
   ndkeywordstyle=\color{violet},           % y las nuevas keywords tendran este color
   frame=simple,                            % simple, sin ningun marco o frame alrededor del codigo
   basewidth={0.55em,0.55em}                % tamano de las letras/lineas. usado para reducir el whitespace entre estas
}

\lstdefinelanguage{http}{
    morekeywords={
        GET,
        PUT,
        POST,
        DELETE
    }
}

\lstdefinestyle{normalhttp}{language=http,  % lenguaje HTTP
   numbers=left,                            % enumerar las lineas
   keywordstyle=\color{darkblue}\textbf,    % color de las keywords
   stringstyle=\color{red},                 % color de los strings
   commentstyle=\color{red},                % color de los comentarios
   basicstyle=\color{black}\ttfamily\footnotesize\bfseries,     % color del texto en general
   morecomment=[l][\color{magenta}]{\#},    % coloreamos las intrucciones del precompilador (todo lo que empieza con #)
   ndkeywords={NULL,nullptr},               % definimos una nuevas keywords como NULL y nullptr
   ndkeywordstyle=\color{violet},           % y las nuevas keywords tendran este color
   frame=simple,                            % simple, sin ningun marco o frame alrededor del codigo
   basewidth={0.55em,0.55em}                % tamano de las letras/lineas. usado para reducir el whitespace entre estas
}

% Pone las constantes numericas con color (violeta en este caso):
% - los numeros que estan dentro de un string/comentarios NO son coloreados
% - los numeros por fuera que pertenecen a una palabra SON coloreados
%   (buf1, por ejemplo, el "1" es coloreado cuando no lo deberias. UPS!!)
% - No incluye el signo
%
%\lstset{literate=%
%  *{0}{{{\color{violet}0}}}1
%   {1}{{{\color{violet}1}}}1
%   {2}{{{\color{violet}2}}}1
%   {3}{{{\color{violet}3}}}1
%   {4}{{{\color{violet}4}}}1
%   {5}{{{\color{violet}5}}}1
%   {6}{{{\color{violet}6}}}1
%   {7}{{{\color{violet}7}}}1
%   {8}{{{\color{violet}8}}}1
%   {9}{{{\color{violet}9}}}1
%}

% Para resaltar lineas de codigo usando \btLstHLB{range} y \btLstHLB<overlay>{range}:
% Por ejemplo,
%  \btLstHLB{3}       linea 3 resaltada
%  \btLstHLB{1-5}     lineas de la 1 a la 5 resaltadas
%  \btLstHLB<2>{3}    linea 3 resaltada solo en el slide 2
%
% \btLstHLB usa un color azul mientras que \btLstHLR usa un color rojo como fondo
\usepackage{lstlinebgrd}
\makeatletter
\newcount\bt@rangea
\newcount\bt@rangeb

\newcommand\btIfInRange[2]{%
   \global\let\bt@inrange\@secondoftwo%
   \edef\bt@rangelist{#2}%
   \foreach \range in \bt@rangelist {%
      \afterassignment\bt@getrangeb%
      \bt@rangea=0\range\relax%
      \pgfmathtruncatemacro\result{ ( #1 >= \bt@rangea) && (#1 <= \bt@rangeb) }%
      \ifnum\result=1\relax%
      \breakforeach%
      \global\let\bt@inrange\@firstoftwo%
      \fi%
   }%
   \bt@inrange%
}
\newcommand\bt@getrangeb{%
   \@ifnextchar\relax%
   {\bt@rangeb=\bt@rangea}%
   {\@getrangeb}%
}
\def\@getrangeb-#1\relax{%
   \ifx\relax#1\relax%
      \bt@rangeb=100000%   \maxdimen is too large for pgfmath
   \else%
      \bt@rangeb=#1\relax%
   \fi%
}

%%%%%%%%%%%%%%%%%%%%%%%%%%%%%%%%%%%%%%%%%%%%%%%%%%%%%%%%%%%%%%%%%%%%%%%%%%%%%%
%
% \btLstHL<overlay spec>{range list}
%
\newcommand<>{\btLstHLB}[1]{%
   \only#2{\btIfInRange{\value{lstnumber}}{#1}{\color{blue!30}}% blue
   {\def\lst@linebgrdcmd####1####2####3{}}}%
}%
\newcommand<>{\btLstHLR}[1]{%
   \only#2{\btIfInRange{\value{lstnumber}}{#1}{\color{red!30}}% red
   {\def\lst@linebgrdcmd####1####2####3{}}}%
}%
%
%
%%%%%%%%%%%%%%%%%%%%%%

\makeatother


% sin fecha
\date{}

\author[7542-9508]{Di Paola Mart\'in \\ \texttt{martinp.dipaola <at> gmail.com} }

\institute[Universidad de Buenos Aires]
{
   Facultad de Ingenier\'ia\\
   Universidad de Buenos Aires
}

% Definimos una imagen para que este en cada slide
%\pgfdeclareimage[height=0.5cm]{university-logo}{imgs/fiuba.png}
%\logo{\pgfuseimage{university-logo}}

%% Definir estas en el documento final
%%
%% \title%
%% {Programaci\'on gen\'erica y templates en C++}
%%
%% \subject{Programaci\'on gen\'erica y templates en C++}


%%%%%
~% if interactive is on %~
    \AtBeginSection{}
    \AtBeginSubsection{}
~% else %~
    % At begin of each Section do nothing; at each Subsection show the Section and Subsection names
    % If the Section doesn't have a Subsection, then YOU must to add a unnumbered Subsection like \subsection*{}
    % so the AtBeginSubsection will get triggered
    \AtBeginSection{}
    \AtBeginSubsection[\frame{\subsectionpage}]{\frame{\subsectionpage}}
~% endif %~
%
%%%%%




\title%
{Programaci\'on orientada a eventos}


\subject{Programaci\'on orientada a eventos}


\begin{document}

\begin{frame}
   \titlepage
\end{frame}


\section{Programaci\'on orientada a eventos}

\begin{frame}[fragile]{Programaci\'on orientada a eventos}
        \begin{tikzpicture}[remember picture,overlay]
            \begin{scope}[scale=0.8, transform shape]
            \node[yshift=-1cm,at=(current page.center)] {
                \includegraphics<1>[width=\paperwidth]{imgs/eventloop.png}
            };
            \end{scope}
        \end{tikzpicture}
\end{frame}
\note[itemize] {
\item M\'ultiple fuentes de eventos: el input del usuario (mouse, teclado) y el paso del tiempo son solo algunos ejemplos.
\item El mismo programa puede generar eventos para que sean procesados luego.
\item Si m\'ultiples events son generados, estos se guardan en una cola de eventos (dependiendo de la librer\'ia los eventos son sacados en el orden en que entraron FIFO o no).
}

\begin{frame}[fragile]{Prohibido usar handlers lentos}
   \begin{lstlisting}[style=normal]
void save_button_handler() {
   FILE *f = fopen("data.txt", "wt");

   /* ... */
   fwrite(data, sizeof(char), data_sz, f);
   /* ... */

   fclose(f);
}
   \end{lstlisting}
\end{frame}
\note[itemize] {
\item Los handlers no deben bloquearse ni realizar tareas que lleven mucho tiempo para no bloquear a todo el programa.
\item Se puede tomar un handler y particionarlo en subpartes para ejecutarlo iterativa e incrementalmente sin bloquear el programa. No veremos esta alternativa en la materia.
}

\begin{frame}[fragile]{Handlers lentos: multithreading}
   \begin{lstlisting}[style=normal]
void save_background() {
   FILE *f = fopen("data.txt", "wt");

   /* ... */
   fwrite(data, sizeof(char), data_sz, f);
   /* ... */

   fclose(f);
}

void save_button_handler() {
   std::thread t1 {save_background};
}
   \end{lstlisting}
\end{frame}
\note[itemize] {
\item Si es necesario, el handler puede lanzar un hilo para hacer la tarea en background.
\item Esto plantea una serie de preguntas: quien hace el \lstinline[style=normal]!join! sobre el hilo? Podr\'ia haber otro hilo haciendo joins (ineficiente) o prodr\'iamos usar m\'as eventos.
}

\begin{frame}[fragile]{Handlers lentos: multithreading - join}
   \begin{lstlisting}[style=normal]
void save_background() {
   FILE *f = fopen("data.txt", "wt");

   /* ... */
   fwrite(data, sizeof(char), data_sz, f);
   /* ... */

   fclose(f);
   emit_event("joinme", thread);
}

void save_button_handler() {
   std::thread t1 {save_background};
}

void joinme_handler(thread) {
    thread.join();
}
   \end{lstlisting}
\end{frame}
\note[itemize] {
\item Al finalizar \lstinline[style=normal]!save_background! emite un evento y un handler registrado hara el \lstinline[style=normal]!join!.
\item El \lstinline[style=normal]!joinme_handler! es llamado al ocurrir el evento y hace el \lstinline[style=normal]!join! sin bloquearse.
}

\begin{frame}[fragile]{B\'usqueda de los handlers (tree version)}
        \begin{tikzpicture}[remember picture,overlay]
            \begin{scope}[scale=0.45, transform shape]
            \node[yshift=-1cm,at=(current page.center)] {
                \includegraphics<1>[width=\paperwidth]{imgs/wp.png}
            };
            \end{scope}
        \end{tikzpicture}
\end{frame}
\note[itemize] {
\item Dado un evento, c\'omo se busca a su handler? Depende de la librer\'ia usada.
\item En las librer\'ias gr\'aficas y en los web browsers, las ventanas y p\'aginas web son representadas por estructuras jer\'arquicas (\'arboles).
}

\begin{frame}[fragile]{B\'usqueda de los handlers (tree version): capture phase}
        \begin{tikzpicture}[remember picture,overlay]
            \begin{scope}[scale=0.45, transform shape]
            \node[yshift=-1cm,at=(current page.center)] {
                \includegraphics<1>[width=\paperwidth]{imgs/wp2.png}
            };
            \end{scope}
        \end{tikzpicture}
\end{frame}
\note[itemize] {
\item En estas estructuras tipo \'arbol un evento puede ser capturado por varios objetos: un click en un texto puede verse como un click en el bot\'on que tiene el texto o como un click en el todo que tiene al bot\'on.
\item En los web browsers la b\'usqueda del handler se hace del todo al espec\'ifico en lo que se conoce como \lstinline[style=normal]!capture phase!.
}

\begin{frame}[fragile]{B\'usqueda de los handlers (tree version): bubble phase}
        \begin{tikzpicture}[remember picture,overlay]
            \begin{scope}[scale=0.45, transform shape]
            \node[yshift=-1cm,at=(current page.center)] {
                \includegraphics<1>[width=\paperwidth]{imgs/wp3.png}
            };
            \end{scope}
        \end{tikzpicture}
\end{frame}
\note[itemize] {
\item En una etapa posterior, el web browser hace una segunda pasada en el orden inverso de lo espec\'ifico a lo general, de ah\'i el nombre \lstinline[style=normal]!bubble phase!.
\item Otras librer\'ias gr\'aficas tienen b\'usquedas similares, otras s\'olo tienen una fase de b\'usqueda. Otras incluso simplemente levantan un bit en un array.
\item En general, un handler puede notificar que el evento debe cancelarse: el resto de los handlers no es llamado. Pero como todo, depender\'a de la librer\'ia usada.
}


\begin{frame}[fragile]{B\'usqueda de los handlers (bit version): select}
   \begin{lstlisting}[style=normal]
fd_set rfds;
struct timeval tv;

while (...) {
   FD_ZERO(&rfds);
   FD_SET(0, &rfds); /* 0 es la entrada estandar */

   /* timeout de 5 segundos */
   tv.tv_sec = 5;
   tv.tv_usec = 0;

   if (select(1, &rfds, NULL, NULL, &tv) == -1)
       perror("select()");
   else if (FD_ISSET(0, &rfds))
       read(0, ...,); /* no deberia bloquearse */
   else
       /* time out */
}
   \end{lstlisting}
\end{frame}
\note[itemize] {
\item En este caso el sistema operativo nos permite ver a los file descriptors como fuentes de eventos: lectura disponible, escritura disponible, excepcion disponible.
\item A diferencia de una librer\'ia gr\'afica, el proceso de b\'usqueda se hace con un array de bits manipulado con las macros \lstinline[style=normal]!FD_ZERO!, \lstinline[style=normal]!FD_SET! y \lstinline[style=normal]!FD_ISSET!
\item \lstinline[style=normal]!select! se bloquea hasta que uno o varios eventos llegan sobre los file descriptors marcados: no hay tal FIFO.
\item En la c\'atedra no se usara \lstinline[style=normal]!select! ni similares pero se deja como ejemplo.
}

\end{document}


