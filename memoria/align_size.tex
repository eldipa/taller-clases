\input{../preamble-tmp.tex}

\title%
{Memoria en C++ - Tama\~no y Alineaci\'on}


\subject{Memoria en C++ - Tama\~no y Alineaci\'on}


\begin{document}

\begin{frame}[noframenumbering,plain]
   \titlepage
\end{frame}


~% macro content() %~
\section{Tama\~nos, Alineaci\'on y Padding}
% sizeof int, char, char*, alineacion y padding
\begin{frame}[fragile,label=RM]{Exacta reserva de memoria}
   \begin{columns}
      \begin{column}{.55\linewidth}
         \begin{lstlisting}[style=dimmided]
@char c = 'A';@@
int i = 1;@@
short int s = 4;@@
char *p = 0;
int *g = 0;@@
int b[2] = {1, 2};@@
char a[] = "AB";@
         \end{lstlisting}
   \begin{itemize}
      \item<1-> Todo depende de la arquitectura y del compilador
      \item<2-> Alineaci\'on y padding
      \item<4-> Punteros del mismo tama\~no
      \item<6-> Un cero como "fin de string"
   \end{itemize}

      \end{column}
      \begin{column}{.05\linewidth}
      \end{column}
      \begin{column}{.30\linewidth}
~% if interactive is off %~
         \begin{tikzpicture}[cell/.style={rectangle,draw=black},
            space/.style={minimum height=1.5em,matrix of nodes,row sep=-\pgflinewidth,column sep=-\pgflinewidth,column 1/.style={font=\ttfamily}},text depth=0.5ex,text height=2ex,nodes in empty cells]

            \matrix (first)[space, column 1/.style={font=\ttfamily},column 2/.style={nodes={cell,minimum width=2em}},column 3/.style={nodes={cell,minimum width=2em}},column 4/.style={nodes={cell,minimum width=2em}},column 5/.style={nodes={cell,minimum width=2em}}]
            {
               {\visible<1->{c}}   & {\visible<1->{65}} & |[fill=gray]| &  |[fill=gray]| &  |[fill=gray]| \\
               {\visible<2->{i}}   & {\visible<2->{0}} & {\visible<2->{0}} & {\visible<2->{0}} & {\visible<2->{1}}\\
               {\visible<3->{s}}   & {\visible<3->{0}} & {\visible<3->{4}} & |[fill=gray]|  & |[fill=gray]| \\
               {\visible<4->{p}}   & {\visible<4->{0}} & {\visible<4->{0}} & {\visible<4->{0}} & {\visible<4->{0}}\\
               {\visible<4->{g}}   & {\visible<4->{0}} & {\visible<4->{0}} & {\visible<4->{0}} & {\visible<4->{0}}\\
               {\visible<5->{b}}   & {\visible<5->{0}} & {\visible<5->{0}} & {\visible<5->{0}} & {\visible<5->{1}}\\
                                   & {\visible<5->{0}} & {\visible<5->{0}} & {\visible<5->{0}} & {\visible<5->{2}}\\
               {\visible<6->{a}}   & {\visible<6->{65}} & {\visible<6->{66}} & {\visible<6->{0}} & |[fill=gray]|\\
            };
         \end{tikzpicture}
~% endif %~
      \end{column}
   \end{columns}
\end{frame}
\note[itemize] {
\item C, C++ y Rust son lenguajes de bajo nivel para que el programador pueda tener un control absoluto de d\'onde y c\'omo se ejecuta el c\'odigo.
\item El tama\~no en bytes de los tipos depende de la arquitectura y del compilador. En \lstinline[style=normal]!\#include <cstdint>! se puede tener acceso a tipos con tama\~nos espec\'ificos como \lstinline[style=normal]!uint64_t!
\item El compilador puede guardar las variables en posiciones de memoria m\'ultiplos de 4 (depende de la arquitectura y de los flags de compilaci\'on): variables alineadas son accedidas m\'as r\'apidamente que las desalineadas.
\item Como contra, la alineaci\'on despedicia espacio (padding) hay un tradeoff entre velocidad y espacio.
\item El tama\~no de un puntero no depende de a que tipo apunta; todos los punteros ocupan el mismo tama\~no (que depende de la arquitectura).
\item A los strings en C/C++ escritos en el c\'odigo del programa el compilador les agrega el caracter nulo (byte 0). Tenerlo en cuenta!!
}


\begin{frame}[fragile]{Agrupaci\'on de variables}
   \begin{columns}
      \begin{column}{.55\linewidth}
         \begin{lstlisting}[style=normal]
struct S {
   int  a;
   char b;
   int  c;
   char d;
};

struct S s = {1,2,3,4};

         \end{lstlisting}
      \end{column}
      \begin{column}{.40\linewidth}
~% if interactive is off %~
         \begin{tikzpicture}[cell/.style={rectangle,draw=black},
            space/.style={minimum height=1.5em,matrix of nodes,row sep=-\pgflinewidth,column sep=-\pgflinewidth,column 1/.style={font=\ttfamily}},text depth=0.5ex,text height=2ex,nodes in empty cells]

            \matrix (first)[space, column 1/.style={font=\ttfamily},column 2/.style={nodes={cell,minimum width=2em}},column 3/.style={nodes={cell,minimum width=2em}},column 4/.style={nodes={cell,minimum width=2em}},column 5/.style={nodes={cell,minimum width=2em}}]
            {
               {s.a}   & {0} & {0} & {0} & {1}\\
               {s.b}   & {2} & |[fill=gray]| &  |[fill=gray]| &  |[fill=gray]| \\
               {s.c}   & {0} & {0} & {0} & {3}\\
               {s.d}   & {4} & |[fill=gray]| &  |[fill=gray]| &  |[fill=gray]| \\
            };
         \end{tikzpicture}
~% endif %~
      \end{column}
   \end{columns}
\end{frame}
\note[itemize] {
\item El padding se hace mas notorio en las estructuras: el acceso a cada atributo es r\'apido pero hay memoria desperdiciada.
}

\begin{frame}[fragile]{Agrupaci\'on de variables}
   \begin{columns}
      \begin{column}{.55\linewidth}
         \begin{lstlisting}[style=normal]
struct S {
   int  a;
   char b;
   int  c;
   char d;
} __attribute__((packed));

struct S s = {1,2,3,4};

         \end{lstlisting}
      \end{column}
      \begin{column}{.45\linewidth}
~% if interactive is off %~
         \begin{tikzpicture}[cell/.style={rectangle,draw=black},
            space/.style={minimum height=1.5em,matrix of nodes,row sep=-\pgflinewidth,column sep=-\pgflinewidth,column 1/.style={font=\ttfamily}},text depth=0.5ex,text height=2ex,nodes in empty cells]

            \matrix (first)[space, column 1/.style={font=\ttfamily},column 2/.style={nodes={cell,minimum width=2em}},column 3/.style={nodes={cell,minimum width=2em}},column 4/.style={nodes={cell,minimum width=2em}},column 5/.style={nodes={cell,minimum width=2em}}]
            {
               {s.a}       & {0} & {0} & {0} & {1}\\
               {s.b/s.c}   & {2} & {0} & {0} & {0}\\
               {s.c/s.d}   & {3} & {4} & |[fill=gray]| & |[fill=gray]|\\
            };
         \end{tikzpicture}
~% endif %~
      \end{column}
   \end{columns}
\end{frame}
\note[itemize] {
\item Con el atributo especial de gcc \lstinline[style=normal]!__attribute__((packed))! el compilador empaqueta los campos sin padding, m\'as eficiente en memoria pero m\'as lento.
\item Y es m\'as lento por que para leer el atributo \lstinline[style=normal]!s.c! hay que hacer 2 lecturas.
\item Y cuidado, en algunas arquitecturas la lectura de atributos desalineados hace crashear al programa!
}

% endianess
\begin{frame}[fragile]{Endianess: representaci\'on en memoria (intro)}
   \begin{columns}
      \begin{column}{.03\linewidth}
         \begin{lstlisting}[style=normalnonumbers]
Hay "537" rupees.



         \end{lstlisting}
      \end{column}
      \begin{column}{.01\linewidth}
      \end{column}
      \begin{column}{.6\linewidth}
         \vspace*{5mm}
         \begin{lstlisting}[style=normalnonumbers]
Hay "quinientos treinta y siete" rupees.
// El digito de la izquierda es
// el *mas* significativo


Hay "setecientos treinta y cinco" rupees.
// El digito de la izquierda es
// el *menos* significativo
         \end{lstlisting}
      \end{column}
   \end{columns}
\end{frame}
\note[itemize] {
\item Por convenci\'on los n\'umeros ar\'abicos se leen de izquierda a derecha con el d\'igito de la izquierda como el m\'as significativo.
\item Otras notaciones podr\'ian tener convenciones distintas.
\item Nota: rupees es la moneda de The Legend of Zelda.
}
\begin{frame}[fragile]{Endianess: representaci\'on en memoria}
   \begin{columns}
      \begin{column}{.03\linewidth}
         \begin{lstlisting}[style=normalnonumbers]
short i = 0x1234;



         \end{lstlisting}
      \end{column}
      \begin{column}{.01\linewidth}
      \end{column}
      \begin{column}{.6\linewidth}
         \vspace*{5mm}
         \begin{lstlisting}[style=normalnonumbers]
((unsigned char*)&i) == {0x12, 0x34}
// Primer byte es el *mas* significativo
// --> arquitectura big endian



((unsigned char*)&i) == {0x34, 0x12}
// Primer byte es el *menos* significativo
// --> arquitectura little endian
         \end{lstlisting}
      \end{column}
   \end{columns}
\end{frame}
\note[itemize] {
\item El byte m\'as significativo se lee/escribe primero (o esta primero en la memoria) en las arquitecturas big endian.
\item Por el contrario en las arquitecturas little endian es el byte menos significativo quien esta primero en la memoria.
\item El endianess es irrelevante si siempre trabajamos los \lstinline[style=normal]!short!s como n\'umeros pero se vuelve relevante en el momento que queremos interpretar un \lstinline[style=normal]!short! como una tira de bytes (\lstinline[style=normal]!char*!) o viceversa. Y esto es necesario cuando queremos escribir un n\'umero en un archivo binario o enviarlo por la red a otra m\'aquina a traves de un socket!
\item Siempre hay que especificar el endianess en que se guardan/envian los datos.
\item Obviamente lo mencionado aqui para los \lstinline[style=normal]!short!s aplica para el resto de los objetos en memoria, como los \lstinline[style=normal]!int!s
}

\begin{frame}[fragile]{Endianess: representaci\'on en memoria}
Se puede cambiar el endianess de una variable \lstinline[style=normal]!short int! y \lstinline[style=normal]!int!
del endianess nativo o "del host" a big endian o "el endianess de la red" y viceversa:
       \begin{itemize}
           \item Host to Network
         \begin{lstlisting}[style=normal]
htons(short int) htonl(int)
        \end{lstlisting}
           \item Network to Host
         \begin{lstlisting}[style=normal]
ntohs(short int) ntohl(int)
        \end{lstlisting}
   \end{itemize}
\end{frame}
\note[itemize] {
\item Para hacer uso de esas funciones hay que hacer \lstinline[style=normal]!\#include <arpa/inet.h>!.
}

~% endmacro %~

~{ content() }~


\end{document}


