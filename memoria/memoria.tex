\input{../preamble-tmp.tex}

\title%
{Memoria en C++}

\subject{Memoria en C++}

\begin{document}

\begin{frame}[noframenumbering,plain]
   \titlepage
\end{frame}

% Present the following BO code but without talking about how the BO works.
% Ask: is it possible to print you win? Very likely they are going to answer no.
% Show that a crafted input indeed gets the you win but do not explain it, leave
% it as an unknown.
% Have a break and then continue with the class without explaining the BO
\begin{frame}[fragile]{Qu\'e input es necesario para obtener un "You win!" ?}
         \begin{lstlisting}[style=normal]
// compilar con flags:
//  -Wno-deprecated-declarations -std=c++11 -fno-stack-protector
#include <cstdio>

int main(int argc, char *argv[]) {
    int cookie = 0;
    char buf[10];

    printf("buf: %08x cookie: %08x\n", buf, &cookie);
    gets(buf);

    if (cookie == 0x41424344) {
        printf("You win!\n");
    }

    return 0;
} // Insecure Programming
         \end{lstlisting}
\end{frame}

\begin{frame}{De qu\'e va esto?}
   \tableofcontents
   % You might wish to add the option [pausesections]
\end{frame}

~% from "align_size.tex" import content as align_size_content with context %~
~{ align_size_content() }~

% Before continuing ask, "why C/C++?" - Very likely the answer will be "speed"
% Show the java_faster_than_c/contar_primos code and show that C is not necessary fast.
% But also show java_faster_than_c/max_segmento ans show that C and Java
% converges to the same run-time. Why is Java getting slower?
% Don't answer and keep the class.

~% from "segments.tex" import content as segments_content with context %~
~{ segments_content() }~

% In the last slide of segments_content there is an explanation of
% cache friendly structures. Combine this and explain java_faster_than_c.

~% from "pointers.tex" import content as pointers_content with context %~
~{ pointers_content() }~

% Now we have all the needed tools to understand a BO: memory segment (stack),
% padding, sizes and endianness. Go for it and explain it!

~% from "buffer_overflow.tex" import content as buffer_overflow_content with context %~
~{ buffer_overflow_content() }~

% The last slide shows a BO that requires smashing the stack further,
% in particular to change the argument of the print from "you loose" to "you win"
% Ask how (they should say the string cannot be changed (code segment)
% but the pointer to the string can (stack) and point to the overflowed buffer.
% Explain now that if the printf() would be a system() you would have full
% control of the machine.

% Pointers notation is optional. Ask if they want it or not.

~% from "pointers-notation.tex" import content as pointers_notation_content with context %~
~{ pointers_notation_content() }~

\appendix
\section<presentation>*{\appendixname}
\subsection<presentation>*{Referencias}

\begin{frame}[allowframebreaks]
   \frametitle<presentation>{Referencias}

   \begin{thebibliography}{10}

         \beamertemplatebookbibitems
         % Start with overview books.

      \bibitem{Stroustrup}
         Bjarne Stroustrup.
         \newblock {\em The C++ Programming Language}.
         \newblock Addison Wesley, Fourth Edition.

         \beamertemplatearticlebibitems
         % Followed by interesting articles. Keep the list short.

      \bibitem{man page: gets strcpy htons qsort}
         man page: gets strcpy htons qsort

      \bibitem{Insecure Programming}
         Insecure Programming

      \bibitem{https://cdecl.org/}
          https://cdecl.org/

      \bibitem{https://www.youtube.com/watch?v=tas0O586t80}
          https://www.youtube.com/watch?v=tas0O586t80


   \end{thebibliography}
\end{frame}

\end{document}


