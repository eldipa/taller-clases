\input{../preamble-tmp.tex}

\title%
{Memoria en C/C++ - Segmentos}

\subject{Memoria en C/C++ - Segmentos}


\begin{document}

\begin{frame}[noframenumbering,plain]
   \titlepage
\end{frame}

~% if interactive is off %~
\begin{frame}{De qu\'e va esto?}
   \tableofcontents
   % You might wish to add the option [pausesections]
\end{frame}
~% endif %~

~% macro content() %~
\section{Segmentos de Memoria}
% Stack Heap Data Code
~% if interactive is off %~
\begin{frame}[fragile,label=SM]{Segmentos de memoria}
   \begin{itemize}
      \item<1-> Code segment: de solo lectura y ejecutable, a donde va el c\'odigo y las constantes.
      \item<2-> Data segment: variables creadas al inicio del programa y son v\'alidas hasta que este termina; pueden ser de acceso global o local.
      \item<3-> Stack: variables creadas al inicio de una llamada a una funci\'on y destruidas autom\'aticamente cuando esta llamada termina.
      \item<4-> Heap: variables cuya duraci\'on esta controlada por el programador (run-time).
   \end{itemize}
\end{frame}
~% else %~
\begin{frame}[fragile]{Segmentos de memoria}
   \begin{itemize}
      \item<1-> Code segment: de solo lectura y ejecutable, a donde va el c\'odigo y las constantes.
      \item<2-> Data segment: variables creadas al inicio del programa y son v\'alidas hasta que este termina; pueden ser de acceso global o local.
   \end{itemize}
\end{frame}
\begin{frame}[fragile]{Segmentos de memoria}
   \begin{itemize}
      \item<3-> Stack: variables creadas al inicio de una llamada a una funci\'on y destruidas autom\'aticamente cuando esta llamada termina.
      \item<4-> Heap: variables cuya duraci\'on esta controlada por el programador (run-time).
   \end{itemize}
\end{frame}
~% endif %~
\begin{frame}[fragile,label=LS]{Duraci\'on y visibilidad (lifetime and scope)}
   \begin{itemize}
       \item<1-> Duraci\'on (lifetime): tiempo desde que a la variable se le reserva memoria hasta que esta es liberada. Determinado por el segmento de memoria que se usa.
       \item<2-> Visibilidad (scope): Cuando una variable se la puede acceder y cuando esta oculta.
   \end{itemize}
\end{frame}


% static (variables locales/globales y funciones)
~% if interactive is off %~
\begin{frame}[fragile]{Asignaci\'on del lifetime y scope}
         \begin{lstlisting}[style=dimmided]
@int g = 1;@@
static int l = 1;@@
extern char e;@
@
void Fa() { }@@
static void Fb() { }@@
void Fc();@

void foo(@int arg@) {@
   int a = 1;@@
   static int b = 1;@
   @
   void * p = malloc(4);
   free(p);@
   @
   char *c = "ABC";@@
   char ar[] = "ABC";@
}
         \end{lstlisting}
\end{frame}
\begin{frame}[fragile]{Asignaci\'on del lifetime y scope}
         \begin{lstlisting}[style=normal]
int g = 1;          // Data segment; scope global
static int l = 1;   // Data segment; scope local (este file)
extern char e;      // No asigna memoria (es un nombre)

void Fa() { }        // Code segment; scope global
static void Fb() { } // Code segment; scope local (este file)
void Fc();           // No asigna memoria (es un nombre)

void foo(int arg) {  // Argumentos y retornos son del stack
   int a = 1;        // Stack segment; scope local (func foo)
   static int b = 1; // Data segment; scope local (func foo)

   void * p = malloc(4); // p en el Stack; apunta al Heap
   free(p);              // liberar el bloque explicitamente!!

   char *c = "ABC";   // c en el Stack; apunta al Code Segment
   char ar[] = "ABC"; // es un array con su todo en el Stack
} // fin del scope de foo: las variables locales son liberadas
         \end{lstlisting}
\end{frame}
~% else %~
\begin{frame}[fragile]{Asignaci\'on del lifetime y scope}
         \begin{lstlisting}[style=normal]
int g = 1;
static int l = 1;
extern char e;

void Fa() { }
static void Fb() { }
void Fc();

void foo(int arg) {
   int a = 1;
   static int b = 1;

   void * p = malloc(4);
   free(p);

   char *c = "ABC";
   char ar[] = "ABC";
}
         \end{lstlisting}
\end{frame}
~% endif %~
\begin{frame}[fragile]{El donde importa!}
         \begin{lstlisting}[style=normal]

void f() {
   char *a = "ABC";
   char b[] = "ABC";

   b[0] = 'X';
   a[0] = 'X';  // segmentation fault
}
         \end{lstlisting}
\end{frame}
\note[itemize] {
\item Como el puntero "a" apunta al Code Segment y este es de solo lectura, tratar de modificarlo termina en un Segmentation Fault
}

~% endmacro %~

~{ content() }~


\end{document}


