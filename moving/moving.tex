
\input{../preamble-tmp.tex}

\title%
{Pasaje de objetos en C++}

\subject{Pasaje de objetos en C++}

\tikzset{
  pointer/.style={->, dashed, draw, black, line width=0.03cm},
}

\tikzset{
 nodeinvisible/.style={opacity=.01},
 nodevisible on/.style={alt={#1{}{nodeinvisible}}},
 arrowinvisible/.style={opacity=.01},
 arrowvisible on/.style={alt={#1{}{arrowinvisible}}},
 alt/.code args={<#1>#2#3}{%
   \alt<#1>{\pgfkeysalso{#2}}{\pgfkeysalso{#3}} % \pgfkeysalso doesn't change the path
 },
}

\begin{document}

\begin{frame}[noframenumbering,plain]
   \titlepage
\end{frame}

\begin{frame}{De qu\'e va esto?}
   \tableofcontents
   % You might wish to add the option [pausesections]
\end{frame}

~% from "moving_types.tex" import content as mv_content with context %~
~{ mv_content() }~

~% from "assign.tex" import content as assign_content with context %~
~{ assign_content() }~

\begin{frame}[fragile]{Take home message}{}
   \begin{itemize}
       \item<1-> En el 99\% de los casos, no queres que tus clases sean copiables (sea por que no tiene sentido o por performance). Deshabilitar el constructor y operado asignaci\'on por copia.
       \item<2-> Implementar siempre move semantics (constructor y operador asignaci\'on).
      \item<3-> Usar const en donde sea posible.
   \end{itemize}
\end{frame}

\appendix
\section<presentation>*{\appendixname}
\subsection<presentation>*{Referencias}

\begin{frame}[allowframebreaks]
   \frametitle<presentation>{Referencias}

   \begin{thebibliography}{10}

         \beamertemplatebookbibitems
         % Start with overview books.

      \bibitem{Stroustrup}
         Bjarne Stroustrup.
         \newblock {\em The C++ Programming Language}.
         \newblock Addison Wesley, Fourth Edition.

   \end{thebibliography}
\end{frame}

\end{document}


